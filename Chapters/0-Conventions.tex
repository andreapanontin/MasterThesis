\section*{Notation and conventions}
Each time we will use the letter $p$ it is going to denote a prime number,
we can fix it here once and for all (tk: I am still unsure whether we'll
need to make it vary during the notes).
With this in mind we will denote by $\mathbb{F}_{p}$ the field with $p$ elements
and by $\mathbb{Z}$ the ring of integers.


With regards to algebraic geometry: for a morphism of schemes $f\colon X \to Y$,
we will denote by $f^{\#}\colon \mathcal{O}_{ Y } \to \mathcal{O}_{ X }$ the
associated morphism between structure sheaves and, given $y \coloneqq f(x)$,
by $f_y^{\#}\colon \mathcal{O}_{ Y,y } \to \mathcal{O}_{ X,x }$ the induced morphism
at the level of stalks.
Finally we denote by $\mathfrak{m}_x$ the maximal ideal of the local ring
$\mathcal{O}_{ X,x }$ and by $\kappa(x) \coloneqq \mathcal{O}_{ X,x }/\mathfrak{m}_x$
the residue field at $x$.

With respect to category theory:
We will not worry about universes. More precisely we will
tacitly assume a universe $\mathbb{U}$ has been chosen
and we restrict to categories whose objects lie in $\mathbb{U}$.
We will denote categories using a sans serif font.
In particular we will use the following notations:
$\mathsf{Sch}_{S}$ for the category of schemes over $S$,
$\mathsf{Gp}$ for that of groups,
$\mathsf{Ab}$ for that of abelian groups,
$\mathsf{Top}$ for that of topological spaces,
and $\mathsf{Sets}$ for that of sets.
Moreover, by {\em ring} or {\em algebra}, we will mean one which is commutative and with unity.
Finally we will often use the shorter $X \in \mathsf{C}$ to mean that $X$ is an object
of the category $\mathsf{C}$, i.e. that $X \in \mathrm{Ob} \left(\mathsf{C}\right)$.
