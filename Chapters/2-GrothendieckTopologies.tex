\documentclass[../Main]{subfiles}
\begin{document}
\section{Grothendieck topologies}
tk: The aim of this section is to introduce
the concept of Grothendieck topology on a category,
in order to describe sheaves on such category.


\begin{defn}[Site]
	Let $\mathsf{C}$ be a category. 
	A {\em pretopology} on $\mathsf{C}$ is the assignement, to each object $U \in \mathsf{C}$
	of a collection of sets of arrows $\left\{ U_{ i } \to U \right\}_{ i \in I }$,
	called {\em covering} of $U$, such that the following conditions are satisfied:
	\begin{enumerate}
		\item Given an isomorphism $V \to U$, the set $\left\{ V \to U \right\}$
			is a covering.
		\item Given a covering $\left\{ U_{ i } \to U \right\}_{ i \in I }$
			of $U$ and a morphism $V \to U$, then the fiber products
			$\left\{ U_i \cross_{ U } V \right\}_{i \in I}$ exist
			and the set of projections
			$\left\{ U_i \cross_{ U } V \to V \right\}_{ i \in I }$
			is a covering of $V$.
		\item Given a covering $\left\{ U_{ i } \to U \right\}_{ i \in I }$
			and, for each index $i \in I$, a covering 
			$\left\{ V_{ ij } \to V_i \right\}_{ j \in J_i }$ of $V_i$,
			the set of composite morphisms
			$\left\{ V_{ ij } \to U \right\}_{ i \in I, j \in J_i }$
			is a covering of $U$.
	\end{enumerate}
	A catogory with a Grothendieck pretopology is called a {\em site}.
\end{defn}


\begin{rem}[]
	From properties $2.$ and $3.$ above it follows that, given 
	$\left\{ U_{ i } \to U \right\}_{ i \in I }$ and $\left\{ V_{ i } \to U \right\}_{ j \in J }$,
	two coverings of the same object, then also
	$\left\{ U_{ i } \cross_U V_j \to U \right\}_{ (i,j) \in I \cross J }$
	is a covering of $U$.
\end{rem}

Let's now give some examples.
In particular, in sets or schemes, we will say that a family of morphisms
$\left\{ U_{ i } \to U \right\}_{ i \in I }$ is {\em jointly surjective} iff
the set-theoretic union of the images is $U$.
\begin{ex}\leavevmode\vspace{-.2\baselineskip}
\begin{enumerate}
	\item The {\em site of a topological space}.
		Let $X$ be a topological space, and $\mathsf{Op}(X)$ denote the category of
		open subsets of $X$.
		Then, to each $U \in \mathsf{Op}(X)$, we associate it the family of all
		open coverings of $U$.
		Noting that, for $U_1 \to U$ and $U_2 \to U$, $U_1 \cross_{ U } U_2 = U_1 \cap U_2$,
		one easily checks that this defines a pretopology on $\mathsf{Op}(X)$.


	\item The {\em global classical topology} on $\mathsf{Top}$.
		Given $U \in \mathsf{Top}$, its coverings are all families of jointly surjective
		collections of open embeddings $U_i \to U$.
		Here open embedding means open, continuous, injective map and not
		only the set theoretic inclusion of a subspace.


%	\item The {\em small {\'e}tale site of a scheme}.
%		Let $X$ be a scheme and $X_{\mathrm{ét}}$ be the full subcategory
%		of $\mathsf{Sch}_{ X }$, with objects $U \in X_{\mathrm{ét}}$ s.t.
%		the structure morphism $U \to X$ is étale, of finite presentation.
%		Then a covering of $U \in X_{\mathrm{ét}}$ is a jointly surjective
%		collection of morphisms $U_i \to U$.

\end{enumerate}
Let's now give a couple of examples in $\mathsf{Sch}_{ S }$, for a fixed scheme $S$.
\begin{enumerate}[resume]
	\item The {\em global Zariski topology}.
		A covering $\left\{ U_{ i } \to U \right\}_{ i \in I }$ is a 
		jointly surjective collection of 
		open embeddings.
		

	\item The {\em global étale topology}.
		A covering $\left\{ U_{ i } \to U \right\}_{ i \in I }$ is a 
		jointly surjective collection of 
		étale maps locally of finite presentation.

	\item The {\em fppf topology}.
		A covering $\left\{ U_{ i } \to U \right\}_{ i \in I }$ is a 
		jointly surjective collection of 
		flat maps locally of finite presentation.
		The acronym fppf stands for 
		"fidèlement plat et de présentation finie".
\end{enumerate}
\end{ex}


\end{document}
