\documentclass[../Main]{subfiles}
\begin{document}
\chapter{Affine group schemes}
In the following we denote by $\mathsf{Sch}_{S}$ the category of schemes over $S$,
by $\mathsf{Gp}$ the category of groups.
\begin{defn}[Group scheme]
	Let $F\colon \mathsf{Sch}_S^{op} \to \mathsf{Gp}$ be a functor.
	If $F$ is representable by $G \in \mathsf{Sch}_{ S }$, i.e.
	\begin{equation}
		F(X) \simeq \mathrm{Hom}_{\mathsf{Sch}_S} \left( X, G \right)
	\end{equation} 
	functorially in $X$, then we call $G$ a {\em group scheme} over $S$.
\end{defn}

\begin{rem}[]
	By definition a group is a set $|G|$, endowed with an operation
	\begin{align}
		m\colon G \cross G &\longrightarrow G \\
		(g,h) &\longmapsto g \cdot h \nonumber
	,\end{align} 
	an inverse map 
	\begin{align}
		\mathrm{inv}\colon G &\longrightarrow G \\
		g &\longmapsto g^{-1} \nonumber
	\end{align}
	and an identity element $e \in G$.
	The above are characterized by the following properties:
	\begin{align}
		g \cdot (h \cdot k) &= (g \cdot h) \cdot k,\\
		g \cdot g^{-1} &= g^{-1} \cdot g = e,\\
		e \cdot g &= g \cdot e = g
	\end{align} 
	for all $g,h,k \in G$.
	These propertiese can be rewritten in terms of the commutativity of some diagrams,
	recalling the definitions of $m, \mathrm{inv}$ and $e$.
	In fact one can view the identity element as a map
	$\varepsilon\colon \left\{ e \right\} \to G$, 
	for $\left\{ e \right\}$ the terminal object in $\mathsf{Gp}$.
	Let's write $\pi\colon G \to \left\{ e \right\}$
	as the unique arrow in the terminal object
	and $\Delta\colon G \to G \cross G$ the diagonal morphism,
	then the above can be rewritten as
	\begin{align}
		m \circ (id_G \cross m) &= m \circ (m \cross id_G),\\
		m \circ (id_G \cross \mathrm{inv}) \circ \Delta &=
		m \circ (\mathrm{inv} \cross id_G) \circ \Delta = \varepsilon \circ \pi,\\
		m \circ (\varepsilon \cross id_G) &=
		m \circ (id_G \cross \varepsilon) = id_G
	.\end{align} 
\end{rem}


\begin{rem}[]
	tk: this construction works in any category admitting finite products (if I'm not mistaken),
	moreover one should stress the fact that fibered product in $\mathsf{Sch}_{ S }$
	are just products, since $S$ is the final object of the category.
	Notice that Yoneda allows us to define, starting from the morphisms in $\mathsf{Gp}$ of $F(X)$,
	the following morphisms in $\mathsf{Sch}_{ S }$:
	\begin{align}
		m\colon G \cross_{ S } G &\longrightarrow G \\
		\varepsilon \colon S &\longrightarrow G \\
		\mathrm{inv}\colon G &\longrightarrow G 
	.\end{align} 
	As done above, one can translate the properties of associativity, identity element
	and inverse into relations among those three maps.
	These maps inherit those properties from the maps of groups, by Yoneda.
	Let's write them down for completeness' sake:
	\begin{enumerate}
		\item Associativity of the product is equivalent to the commutativity of
			\begin{equation}
			\begin{tikzcd}[column sep=4.5em]
				G \cross_{ S } G \cross_{ S } G 
				\arrow[r, "id_G \cross_S m", rightarrow] 
				\arrow[d, "m \cross_S id_G"', rightarrow] &
				G \cross_{ S } G \arrow[d, "m", rightarrow] \\
				G \cross_{ S } G \arrow[r, "m"', rightarrow] &
				G.
			\end{tikzcd}
			\end{equation} 
		\item The identity element is characterized by the following two diagrams
			(one for the left identity and one for the right identity, which coincide)
			\begin{equation}
			\begin{tikzcd}
				G \cross_{ S } G \arrow[r, "m", rightarrow] &
				G \\
				S \cross_{ S } G \arrow[r, "id_G", rightarrow] 
				\arrow[u, "\varepsilon \cross_S id_G", rightarrow] &
				G \arrow[u, "id_G"', rightarrow] 
			\end{tikzcd}
			\qquad
			\begin{tikzcd}
				G \cross_{ S } G \arrow[r, "m", rightarrow] &
				G \\
				G \cross_{ S } S \arrow[r, "id_G", rightarrow] 
				\arrow[u, "id_G \cross_S \varepsilon", rightarrow] &
				G. \arrow[u, "id_G"', rightarrow] 
			\end{tikzcd}
			\end{equation} 
		\item Finally the inverse morphism is characterized by the following
			commutative diagrams (again, one for being the left inverse, 
			the other one for the left inverse, and they coincide)
			\begin{equation*}
			\begin{tikzcd}[column sep=2.7em]
				G \cross_{ S } G \arrow[rr, "id_G \cross_S \mathrm{inv}", rightarrow] & &
				G \cross_{ S } G \arrow[d, "m", rightarrow] \\
				G \arrow[u, "\Delta", rightarrow] 
				\arrow[r, "\pi", rightarrow] &
				S \arrow[r, "\varepsilon", rightarrow] & 
				G
			\end{tikzcd}
			\qquad
			\begin{tikzcd}[column sep=2.7em]
				G \cross_{ S } G \arrow[rr, "\mathrm{inv} \cross_S id_G", rightarrow] & &
				G \cross_{ S } G \arrow[d, "m", rightarrow] \\
				G \arrow[u, "\Delta", rightarrow] 
				\arrow[r, "\pi", rightarrow] &
				S \arrow[r, "\varepsilon", rightarrow] & 
				G
			\end{tikzcd}
			\end{equation*} 
	\end{enumerate}
\end{rem}

\begin{prop}[]\leavevmode\vspace{-.2\baselineskip}
	\begin{enumerate}
		\item 
	Consider a pullback diagram in $\mathsf{Sch}_{ S }$
	\begin{equation}
	\begin{tikzcd}
		P \arrow[r, "p_1", rightarrow] \arrow[d, "p_2"', rightarrow] &
		Y \arrow[d, "g", rightarrow] \\
		X \arrow[r, "f", rightarrow] &
		Z
	\end{tikzcd}
	.\end{equation} 
	Assume that $g$ is an open (resp. closed) immersion, then $p_2$ is.
	\item Let $f\colon X \to Y$ be a separated morphism of schemes.
		Consider $s\colon Y \to X$ a section of $f$.
		Then $s$ is a closed immersion.
	\item An $S$-group scheme $G$ is separated iff the unit section $\varepsilon$
		is a closed immersion.
	\end{enumerate}
\end{prop}
\begin{proof}\leavevmode\vspace{-.2\baselineskip}
\begin{enumerate}
\item Notice that, by definition of pullback and fiber product, a pullback is just
		a diagram like:
	\begin{equation}
	\begin{tikzcd}
		X \cross_{ Z } Y \arrow[r, "p_1", rightarrow] \arrow[d, "p_2"', rightarrow] &
		Y \arrow[d, "g", rightarrow] \\
		X \arrow[r, "f", rightarrow] &
		Z
	\end{tikzcd}
	,\end{equation}
	where $p_1$ and $p_2$ are the projection morphisms.
	Then the claim follows from stability of open and closed immersions under base change, 
	since one can notice that $p_2$ is just $g$ base changed to $f\colon X \to Z$.

\item By the above it is enough to show that the following diagram is a pullback
	\begin{equation}
	\begin{tikzcd}[column sep=4.6em]
		X \arrow[r, "s", rightarrow] \arrow[d, "s"', rightarrow] &
		Y \arrow[d, "\Delta_{Y/X}", rightarrow] \\
		Y \arrow[r, "id_{Y} \cross {(s \circ f)}"', rightarrow] &
		Y \cross_{ X } Y
	\end{tikzcd}
	.\end{equation} 
	One can use Yoneda to prove this.
	tk: do it.

\item Point $2.$ shows the direct implication.
	Moreover, still using Yoneda and proving the statement for $T$ points,
	one can show that the following is a pullback, and conclude by $1.$:
	\begin{equation}
	\begin{tikzcd}[column sep=5.4em]
		G \arrow[r, "\pi", rightarrow] \arrow[d, "\Delta_{G/S}"', rightarrow] &
		S \arrow[d, "\varepsilon", rightarrow] \\
		G \cross_{ S } G \arrow[r, "m \circ {(id_G \cross \mathrm{inv})}"', rightarrow] &
		G.
	\end{tikzcd}\qedhere
	\end{equation} 
\end{enumerate}
\end{proof}

In the following we'll tend to focus on {\em separated} groups schemes over $S$.
Notice that in that case, since $\varepsilon$ is a closed immersion, it is also an affine morphism,
hence as soon as $G$ is affine, also $S$ is.
\begin{rem}[]
	Then, if we assume that $G = \mathrm{Spec}(A)$ is affine, we can translate the above $S$-scheme
	morphisms, for $S = \mathrm{Spec}(R)$ affine, into $R$-algebra morphisms.
	Then the properties defined by diagrams (tk: references) will translate into properties
	for these new morphisms.
	Recall that the structural morphism $\pi\colon G \to S$ corresponds to a
	morphism $R \to A$ making $A$ into an $R$-algebra.
	Moreover the diagonal morphism $\Delta\colon G \to G \cross_{ S } G$ corresponds
	to the codiagonal morphism of $R$-algebras:
	\begin{align}
		\widetilde{\Delta}\colon A \otimes_R A &\longrightarrow A \\
		a \otimes_R b &\longmapsto a \cdot b \nonumber
	.\end{align} 
	Then, one defines the comultiplication, the coinverse, and counit, as the morphisms
	\begin{align}
		\widetilde{m}\colon A &\longrightarrow A \otimes_R A \\
		\widetilde{\mathrm{inv}}\colon A &\longrightarrow A \\
		\widetilde{\varepsilon}\colon A &\longrightarrow R
	.\end{align} 
	Using the contravariant equialence of categories between affine schemes and
	rings, we can translate the properties satisfied by $m, \varepsilon$ and $\mathrm{inv}$
	into analogous for $\widetilde{m}, \widetilde{\varepsilon}$ and $\widetilde{\mathrm{inv}}$:
	\begin{enumerate}
		\item Associativity of the product induces
			\begin{equation}
			\begin{tikzcd}[column sep=4em]
				A \arrow[r, "\widetilde{m}", rightarrow] 
				\arrow[d, "\widetilde{m}"', rightarrow] &
				A \otimes_R A \arrow[d, "\widetilde{m} \otimes_R id_A", rightarrow] \\
				A \otimes_R A \arrow[r, "id_A \otimes_R \widetilde{m}"', rightarrow] &
				A \otimes_R A \otimes_R A.
			\end{tikzcd}
			\end{equation} 
		\item The diagram for the identity induces the following two diagrams
			\begin{equation}
			\begin{tikzcd}
				A \arrow[r, "i_A", rightarrow] &
				A \otimes_R R \\
				A \arrow[r, "\widetilde{m}"', rightarrow] 
				\arrow[u, "id_A", rightarrow] &
				A \otimes_R A \arrow[u, "id_A \otimes_R \widetilde{\varepsilon}"', rightarrow]
			\end{tikzcd}
			\qquad
			\begin{tikzcd}
				A \arrow[r, "i_A", rightarrow] &
				A \otimes_R R \\
				A \arrow[r, "\widetilde{m}"', rightarrow] 
				\arrow[u, "id_A", rightarrow] &
				A, \otimes_R A \arrow[u, "\widetilde{\varepsilon} \otimes_R id_A"', rightarrow]
			\end{tikzcd}
			\end{equation} 
			where the maps $id_A \otimes_R \widetilde{\varepsilon}$
			is induced by dualizing the diagram
			\begin{equation}
			\begin{tikzcd}
				G \arrow[r, "id_G", rightarrow] &
				G\\
				G \cross_{ S } S
				\arrow[u, "p_1", rightarrow] 
				\arrow[d, "p_2"', rightarrow] 
				\arrow[r, "\exists !\, ", dashrightarrow] &
				G \cross_{ S } G
				\arrow[u, "p_1"', rightarrow] 
				\arrow[d, "p_2", rightarrow] \\
				S \arrow[r, "\varepsilon", rightarrow] &
				G.
			\end{tikzcd}
			\end{equation} 
			This gives rise to the commutative diagram
			\begin{equation}
			\begin{tikzcd}
				A \arrow[r, "id_A", rightarrow] 
				\arrow[d, "i_1"', rightarrow] &
				A \arrow[d, "i_1", rightarrow] \\
				A \otimes_R A \arrow[r, "\exists !\, ", dashrightarrow] &
				A \otimes_R R\\
				A \arrow[u, "i_2", rightarrow] 
				\arrow[r, "\widetilde{\varepsilon}", rightarrow] &
				R. \arrow[u, "i_2"', rightarrow] 
			\end{tikzcd}
			\end{equation} 
			In both diagrams the dashed arrows are defined by universal property
			of (co)product in the respective categories (i.e. $\mathsf{Sch}_{ S }$
			and $R$-$\mathsf{Alg}$, the category of $R$-algebras).
		\item Finally the coinverse morphism is characterized by the following:
			\begin{equation}
			\begin{tikzcd}[column sep=5em]
				A \arrow[rr, "\widetilde{\varepsilon}", rightarrow] 
				\arrow[d, "\widetilde{m}"', rightarrow] & &
				R \arrow[d, "", rightarrow] \\
				A \otimes_R A 
				\arrow[r, "id_A \otimes_R \widetilde{\mathrm{inv}}"', rightarrow] &
				A \otimes_R A \arrow[r, "\widetilde{\Delta}"', rightarrow] &
				A,
			\end{tikzcd}
			\end{equation} 
			with an obvious analogue for the left inverse.
			Again, the map $id_A \otimes_R \widetilde{\mathrm{inv}}$
			is defined as in the previous point.
	\end{enumerate}
\end{rem}

\begin{defn}[Hopf algebra]
	An {\em Hopf algebra} is an $R$-algebra $A$,
	endowed with a comultiplication, a couint and a coinverse
	\begin{align}
		\widetilde{m}\colon A &\longrightarrow A \otimes_R A \\
		\widetilde{\mathrm{inv}}\colon A &\longrightarrow A \nonumber\\
		\widetilde{\varepsilon}\colon A &\longrightarrow R \nonumber
	,\end{align} 
	satisfying the conditions given by the three commutative diagrams above, i.e.
	\begin{align}
		( \widetilde{m} \otimes_R id_A ) \circ \widetilde{m} &=
		( id_A \otimes_R \widetilde{m} ) \circ \widetilde{m},\\
		( id_A \otimes_R \widetilde{\varepsilon} ) \circ \widetilde{m} &=
		( \widetilde{\varepsilon} \otimes_R id_A ) \circ \widetilde{m} =
		id_A,\\
		\widetilde{\Delta} \circ ( id_A \otimes_R \widetilde{\mathrm{inv}} ) 
		\circ \widetilde{m} &=
		\widetilde{\Delta} \circ ( \widetilde{\mathrm{inv}} \otimes_R id_A ) 
		\circ \widetilde{m} =
		(R \to A) \circ \widetilde{\varepsilon}
	.\end{align} 
\end{defn}

\begin{rem}[]
	tk: add a remark/proposition on how to translate the group defining morphisms
	into Hopf algebra morphism and viceversa.
\end{rem}


\begin{ex}[Some interesting examples]
	In all of the following examples we assume, for simplicity, that $S = \mathrm{Spec}(R)$
	and $G = \mathrm{Spec}(A)$ are affine schemes.
	\begin{enumerate}
		\item The {\em additive group scheme}, denoted by $\mathbb{G}_a$, is defined by
			\begin{align}
				\mathbb{G}_a\colon \mathsf{Sch}_{ S } &\longrightarrow \mathsf{Gp} \\
				X &\longmapsto \Gamma \left( X , \mathcal{O}_{ X } \right) \nonumber
			,\end{align} 
			in which $\Gamma \left( X , \mathcal{O}_{ X } \right)$ is viewed
			as an additive group.
			Then $\mathbb{G}_a$ is represented by 
			\begin{equation}
				G = \mathrm{Spec}(R[x])
			.\end{equation} 
			In fact, from the characterization of morphisms into affine schemes,
			we obtain that
			\begin{equation}
			\mathrm{Hom}_{\mathsf{Sch}_S} \left( X, G \right) \simeq
			\mathrm{Hom}_{R \text{-}\mathsf{Alg}} 
			\left( R[x], \Gamma \left( X , \mathcal{O}_{ X } \right) \right) \simeq
			\Gamma \left( X , \mathcal{O}_{ X } \right)
			,\end{equation} 
			functorially in $X$.
			But then an $R$-algebra morphism 
			$R[x] \to \Gamma \left( X , \mathcal{O}_{ X } \right)$
			is determined by the image of $x$, hence
			the last isomorphism above.

			Now we can work to translate the morphisms $m, \varepsilon$ and $\mathrm{inv}$
			into Hopf algebra morphisms.
			Let's start by $m$:
			given $f,g \in \mathrm{Hom}_{\mathsf{Sch}_S} \left( X, G \right)$,
			they correspond to 
			\begin{equation}
			\varphi, \psi \in
			\mathrm{Hom}_{R \text{-}\mathsf{Alg}} 
			\left( R[x], \Gamma \left( X , \mathcal{O}_{ X } \right) \right)
			.\end{equation} 
			In particular each of the two can be identified to the element 
			$s \in \Gamma \left( X , \mathcal{O}_{ X } \right)$ to which
			it maps $x \in R[x]$.
			Then we define $f + g$ to be the morphism of schemes corresponding
			to $\varphi + \psi$, i.e. the morphism given by 
			\begin{equation}
				\left( \varphi + \psi \right)(x) :=
				\varphi (x) + \psi(x)
			.\end{equation} 
			Now, by definition $f + g := m \circ (f,g)$.
			Then we want to define $\widetilde{m}$ in such a way that
			\begin{equation}
			(\varphi, \psi) \circ \widetilde{m} = \varphi + \psi
			.\end{equation} 
			In particular we need only specify $\widetilde{m}(x)$ to identify our morphism,
			moreover as soon as we identify a morphism satisfying the above condition,
			$\widetilde{m}$ has to coincide with it by the bijectivity of the correspondance
			between morphism of $R$-algebras and $S$-schemes.
			We can easily check that $\widetilde{m}(x) := x \otimes_R 1 + 1 \otimes_R x$
			does the job, in fact:
			\begin{equation}
				(\varphi, \psi) \circ \widetilde{m} (x) =
				\varphi(x) \cdot 1 + 1 \cdot \psi(x)
			.\end{equation} 
			If we concentrate on $\mathrm{inv}$ now, we see that
			$\mathrm{inv}\, f$ is the morphism of schemes associated to
			$-\varphi$, where $\varphi$ is the morphism of $R$-algebras associated
			to $f$.
			Then, from the above we obtain that $\widetilde{\mathrm{inv}}$
			must satisfy $-\varphi = \varphi \circ \widetilde{\mathrm{inv}}$.
			And, by linearity, we see that $\widetilde{\mathrm{inv}}(x) := -x$ does the job.

			Analogously for the zero element, we know that $m \circ (f, \varepsilon) = f$
			for all $f$.
			Then, $\widetilde{m}$ has to satisfy $(\varphi, \widetilde{\varepsilon}) \circ
			\widetilde{m} = \varphi$ for all $\varphi$.
			But then, on $x$, it acts by
			\begin{equation}
				(\varphi, \widetilde{\varepsilon}) \circ \widetilde{m} (x) =
				\varphi(x) + \widetilde{\varepsilon}(x)
			.\end{equation} 
			And this can happen for all $\varphi$ iff $\widetilde{\varepsilon}(x) = 0$.

			Then we have obtained the explicit description of the {\em Hopf algebra}
			morphisms:
			\begin{align}
				\widetilde{m}(x) &= x \otimes_R 1 + 1 \otimes_R x,\\
				\widetilde{\varepsilon}(x) &= 0, \nonumber\\
				\widetilde{\mathrm{inv}}(x) &= -x \nonumber
			.\end{align} 

		\item The {\em general linear group scheme}, denoted by $\mathbb{GL}_n$,
			is defined by
			\begin{align}
				\mathbb{GL}_n\colon \mathsf{Sch}_{ S } &\longrightarrow \mathsf{Gp} \\
				X &\longmapsto \mathbb{GL}(n)(X) \nonumber
			,\end{align} 
			where $\mathbb{GL}(n)(X)$ is the set of invertible $n \cross n$
			matrices with coefficients in $\Gamma \left( X , \mathcal{O}_{ X } \right)$,
			viewed as a multiplicative group.
			Let's recall that, given $\left( x_{i,j} \right)_{i,j = 1}^n, 
			\left( y_{i,j} \right)_{i,j = 1}^n \in \mathbb{GL}(n)(X)$,
			their product is given by $\left( c_{i,j} \right)_{i,j = 1}^n$,
			where
			\begin{equation}
			c_{i,j} = \sum_{l=1}^{n} x_{i,l} y_{l,j}
			.\end{equation} 
			Then $\mathbb{GL}_n$ is represented by the $S$-scheme
			$\mathrm{Spec}(A)$, where
			\begin{equation}
				A = \frac{R[x_{i,j},y]_{i,j = 1}^n}{(y \cdot D(x_{1,1}, \ldots, x_{n,n})-1)}
			\end{equation} 
			and $D(x_{1,1}, \ldots, x_{n,n})$ is the polynomial for the determinant
			of a $n \cross n$ matrix.
			Reasoning as before, any $f \in \mathrm{Hom}_{\mathsf{Sch}_S}
			\left( X, G \right)$ is just the data of an invertible $n \cross n$ matrix
			with coefficients in $\Gamma \left( X , \mathcal{O}_{ X } \right)$
			(invertibility is granted by adding the indeterminate $y$).
			Then, considered two such functions $f,g$, with associated functions
			$\varphi, \psi$ acting on hte generators of the $R$-algebra
			respectively as
			\begin{align}
				x_{i,j} &\longmapsto a_{i,j}\\
				x_{i,j} &\longmapsto b_{i,j}\nonumber
			.\end{align} 
			Their product is the function $\varphi \cdot \psi$
			corresponding to the product of the matrices, i.e. acting on generators as
			\begin{equation}
				x_{i,j} \longmapsto c_{i,j} := 
				\sum_{l=1}^{n} a_{i,l} b_{l,j}
			.\end{equation} 
			As before we want $\varphi \cdot \psi = (\varphi, \psi) \circ \widetilde{m}$.
			And it is clear that the map
			\begin{equation}
				\widetilde{m}\colon x_{i,j} \longmapsto 
				\sum_{l=1}^{n} x_{i,l} \otimes_R x_{l,j}
			\end{equation} 
			satisfies our request.
	\end{enumerate}
\end{ex}


\end{document}
