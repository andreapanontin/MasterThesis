\documentclass[../Main]{subfiles}
\begin{document}
\section{Divided powers, exponentials and crystals}
\begin{defn}[Ideal with divided powers]\label{defn:dividedPowers}
	Let $A$ be a ring, and $I \triangleleft A$ an ideal of $A$.
	We say that $I$ is equipped with {\em divided powers} iff
	we are given a family of mappings $\left\{ \gamma_n \right\}_{n \geq 1}$,
	where $\gamma_n\colon I \to A$ for all $n \in \mathbb{N}$,
	satisfying, for all $\lambda \in A$ and $x,y \in I$, the following conditions:
	\begin{enumerate}
		\item $\gamma_0(x) = 1, \gamma_1(x) = x, 
			\gamma_n(x) \in I$ for all $n \geq 2$;
		\item $\gamma_n(\lambda x) = \lambda^n \gamma_n(x)$;
		\item $\gamma_n(x) \cdot \gamma_m(x) =
			\frac{\left( m + n \right)!}{m! n!} \gamma_{m+n}(x)$;
		\item $\gamma_n(x+y) = \sum_{i=0}^{n} \gamma_{n-i}(x) \gamma_i(y)$;
		\item $\gamma_m(\gamma_n(x)) = 
			\frac{\left( mn \right)!}{\left( n! \right)^m m!} \gamma_{mn}(x)$.
	\end{enumerate}
	Given such a system, we say that $(I,\gamma)$
	is an {\em ideal with divided powers},
	where we denoted $\gamma \coloneqq \left\{ \gamma_n \right\}_{n \in \mathbb{N}}$.
	Moreover we might sometimes use the notation
	$x^{(n)} \coloneqq \gamma_n(x)$.
	Finally, to stress the ring we are working in, we might write
	$\left(A, I, \gamma\right)$ to denote $I \triangleleft A$
	an ideal with divided powers given by $\gamma$.
\end{defn}


\begin{defn}[Nilpotent divided powers]\label{defn:NilpotentDividedPowers}
	Given $\left(A, I, \gamma\right)$ as before, we say that the divided
	powers are nilpotent iff there is $N \in \mathbb{N}$ such that,
	for all $i_1 + \cdots + i_k \geq N$,
	the ideal generated by elements of the form
	\begin{equation*}
		\gamma_{i_1}(x_1) \cdot \cdots \cdot \gamma_{i_n}(x_k)
	\end{equation*}
	is zero.
\end{defn}


\begin{rem}[]
	We can make the following easy remarks.
\begin{enumerate}
\item Axiom $2$ of \cref{defn:dividedPowers}
	implies that $\gamma_n(0) = 0$ for all $n \in \mathbb{N}$.

\item Axioms $1$ and $3$ tell us that
	$n! \gamma_n(x) = x^n$.

\item Reasoning by induction one can show that
	\begin{equation*}
		\frac{\left( mn \right)!}{\left( n! \right)^m m!} =
		\prod_{i=1}^{m-1} \frac{\left( in + n - 1 \right)!}{(in)! (n-1)!}
	,\end{equation*}
	which shows that it is an integer.
	In fact it can be interpreted as the number of partition of a set with $mn$
	elements into $m$ subsets of $n$ elements each.

\item In \cref{defn:NilpotentDividedPowers},
	if we take $k = N$ and $i_1 = \ldots i_N = 1$,
	then, thanks to axiom $1$ of \cref{defn:dividedPowers}, the ideal $I$
	is nilpotent.
	In particular $I^N = (0)$.
\end{enumerate}
\end{rem}


\begin{ex}\leavevmode\vspace{-.2\baselineskip}
\begin{enumerate}
	\item Given any ring $A$, $(0)$ is an ideal with divided powers,
		with $\gamma_n(0) = 0$ for all $n \in \mathbb{N}$.

	\item If $A$ is a $\mathbb{Q}$-algebra, every ideal has a unique
		divided powers structure, given by $\gamma_n(x) = x^n/n!$.

	\item If $V$ is a discrete valuation ring of unequal characteristic $p$
		and uniformizer $\pi$, we can write $p = u \pi^e$,
		where $u$ an invertible element and $e$ the absolute ramification
		index of $V$.
		Then $\left( \pi \right)$ has a divided powers structure iff $e \leq p-1$.
		In such case $\gamma$ is unique, determined by
		$\gamma_n(x) \coloneqq x^n/n!$.
		tk: finish this example
\end{enumerate}
\end{ex} 


\begin{defn}[Morphism of divided powers]
	A morphism of divided powers $u\colon \left(A, I, \gamma\right) \to \left(B, J, \delta\right)$
	is a ring homomorphism such that
	$u(I) \subset J$ and, for all $x \in I$, $u(\gamma_n(x)) = \delta_n(u(x))$.
\end{defn}


\end{document}
