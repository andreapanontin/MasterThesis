\documentclass[../Main]{subfiles}
\begin{document}
\section{Divided powers, exponentials and crystals}
\begin{defn}[Ideal with divided powers]\label{defn:dividedPowers}
	Let $A$ be a ring, and $I \triangleleft A$ an ideal of $A$.
	We say that $I$ is equipped with {\em divided powers} iff
	we are given a family of mappings $\left\{ \gamma_n \right\}_{n \geq 1}$,
	where $\gamma_n\colon I \to A$ for all $n \in \mathbb{N}$,
	satisfying, for all $\lambda \in A$ and $x,y \in I$, the following conditions:
	\begin{enumerate}
		\item $\gamma_0(x) = 1, \gamma_1(x) = x$ and 
			$\gamma_n(x) \in I$ for all $n \geq 2$;
		\item $\gamma_n(\lambda x) = \lambda^n \gamma_n(x)$;
		\item $\gamma_n(x) \cdot \gamma_m(x) =
			\frac{\left( m + n \right)!}{m! n!} \gamma_{m+n}(x)$;
		\item $\gamma_n(x+y) = \sum_{i=0}^{n} \gamma_{n-i}(x) \gamma_i(y)$;
		\item $\gamma_m(\gamma_n(x)) = 
			\frac{\left( mn \right)!}{\left( n! \right)^m m!} \gamma_{mn}(x)$.
	\end{enumerate}
	Given such a system, we say that $(I,\gamma)$
	is an {\em ideal with divided powers},
	where we denoted $\gamma \coloneqq \left\{ \gamma_n \right\}_{n \in \mathbb{N}}$.
	Moreover we might sometimes use the notation
	$x^{(n)} \coloneqq \gamma_n(x)$.
	Finally, to stress the ring we are working in, we might write
	$\left(A, I, \gamma\right)$ to denote $I \triangleleft A$
	an ideal with divided powers given by $\gamma$, and 
	we will refer to it as a {\em ring with divided powers}.
\end{defn}


\begin{defn}[Nilpotent divided powers]\label{defn:NilpotentDividedPowers}
	Given $\left(A, I, \gamma\right)$ as before, we say that the divided
	powers are nilpotent iff there is $N \in \mathbb{N}$ such that,
	for all $i_1 + \cdots + i_k \geq N$,
	the ideal generated by elements of the form
	\begin{equation*}
		\gamma_{i_1}(x_1) \cdot \cdots \cdot \gamma_{i_n}(x_k)
	\end{equation*}
	is zero.
\end{defn}


\begin{rem}[]
	We can make the following easy remarks.
\begin{enumerate}
\item Axiom $2$ of \cref{defn:dividedPowers}
	implies that $\gamma_n(0) = 0$ for all $n \in \mathbb{N}$.

\item Axioms $1$ and $3$ tell us that
	$n! \gamma_n(x) = x^n$.

\item Reasoning by induction one can show that
	\begin{equation*}
		\frac{\left( mn \right)!}{\left( n! \right)^m m!} =
		\prod_{k=1}^{m-1} \frac{\left( kn + n - 1 \right)!}{(kn)! (n-1)!}
	,\end{equation*}
	which shows that it is an integer.
	In fact it can be interpreted as the number of partition of a set with $mn$
	elements into $m$ subsets of $n$ elements each.

\item In \cref{defn:NilpotentDividedPowers},
	if we take $k = N$ and $i_1 = \ldots i_N = 1$,
	then, thanks to axiom $1$ of \cref{defn:dividedPowers}, the ideal $I$
	is nilpotent.
	In particular $I^N = (0)$.
\end{enumerate}
\end{rem}


\begin{ex}\leavevmode\vspace{-.2\baselineskip}
\begin{enumerate}
	\item Given any ring $A$, $(0)$ is an ideal with divided powers,
		with $\gamma_n(0) = 0$ for all $n \in \mathbb{N}$.

	\item If $A$ is a $\mathbb{Q}$-algebra, every ideal has a unique
		divided powers structure, given by $\gamma_n(x) = x^n/n!$.

	\item If $V$ is a discrete valuation ring of unequal characteristic $p$
		and uniformizer $\pi$, we can write $p = u \pi^e$,
		where $u$ an invertible element and $e$ the absolute ramification
		index of $V$.
		Then $\left( \pi \right)$ has a divided powers structure iff $e \leq p-1$.
		In such case $\gamma$ is unique, determined by
		$\gamma_n(x) \coloneqq x^n/n!$.
		In fact it is known that, denoted by $\nu_p$ the valuation
		of $\mathbb{Z}_{p}$ normalized to have value group $\mathbb{Z}$,
		and if $n = a_0 + \cdots + a_kp^k$ is the $p$-adic expansion
		of $n \in \mathbb{N}$, then
		\begin{equation*}
			\nu_p(n!) = 
			\frac{n - s_p(n)}{p-1}
		,\end{equation*}
		where $s_p(n) = a_0 + \cdots a_k$.
		Then, in order for $\gamma_n$ to all have values in $(\pi)$
		we need that $\nu(\gamma_n(x)) > 0$ for all $x \in (\pi)$.
		By axiom $2$ of \cref{defn:dividedPowers}
		it is enough to check it for $\pi$.
		Then, assuming $\nu$ is normalized to have values in $\mathbb{Z}$,
		we have
		\begin{equation*}
			\nu(\gamma_n(\pi)) = \nu(\pi^n/n!) =
			n - e \cdot \nu_p(n!) =
			n \cdot \frac{p - 1 - e}{p - 1} + e \cdot \frac{s_p(n)}{p - 1}
		.\end{equation*}
		It is clear that $\nu(\gamma_n(\pi)) > 0$ for all $n \in \mathbb{N}$
		iff $p - 1 - e \geq 0$ iff $e \leq p - 1$.
\end{enumerate}
\end{ex} 


\begin{defn}[Morphism of divided powers]
	A morphism of divided powers, denoted by
	\begin{equation*}
	\begin{tikzcd}[row sep = 0ex
		,/tikz/column 1/.append style={anchor=base east}
		,/tikz/column 2/.append style={anchor=base west}]
		u\colon \left(A, I, \gamma\right) 
		\arrow[r, "", rightarrow] &
		\left(B, J, \delta\right)
	,\end{tikzcd}
	\end{equation*} 
	is a ring homomorphism $u\colon A \to B$ such that
	$u(I) \subset J$ and $u(\gamma_n(x)) = \delta_n(u(x))$
	for all $x \in I$.
\end{defn}

\noindent
Let's now introduce the analogous of the symmetric algebra 
in the divided powers context:
\begin{thm}[Bertelot \S3, theorem 3.9]
	Let $M$ be an $A$-module.
	Then there exists a divided powers algebra 
	$\left(\Gamma_A(M), \Gamma_A^+(M), \gamma\right)$ and an $A$-linear
	map $\varphi\colon M \to \Gamma_A^+(M)$ with the following
	universal property:
	given any other divided powers algebra
	$\left(B, J, \delta\right)$ and any $A$-linear
	map $\psi\colon M \to J$,
	then there is a unique divided powers morphism
	$\overline{\psi}\colon \Gamma_A(M) \to B$
	such that $\overline{\psi} \circ \varphi = \psi$.
	Moreover the divided powers algebra $\Gamma(M)$ satisfies:
\begin{enumerate}
	\item $\Gamma^+(M)$ is a graded algebra, with
		$\Gamma_A^0(M) = A$, $\Gamma_A^1(M) = M$
		and $\Gamma_A^+(M) = \oplus_{i \geq 1} \Gamma_A^i(M)$.

	\item The functor $M \mapsto \Gamma_A(M)$ is compatible with
		base change $A \to A'$, i.e. given any $A$algebra $A'$
		then
		\begin{equation*}
			\Gamma_{A'}(M \otimes_A A') \simeq A' \otimes_A \Gamma_A(M)
		.\end{equation*}

	\item Given a direct system of $A$-modules $\left\{ M_j \right\}_{j \in J}$ 
		and denoted $M = \varinjlim_j M_j$, then $\Gamma_A(M) = \varinjlim_{j} \Gamma_A(M_j)$.

	\item For any pair of $A$-algebras $M, N$, we have $\Gamma_A(M \oplus N)
		\simeq \Gamma_A(M) \otimes \Gamma_A(N)$.

	\item Denote $x^{(1)} \coloneqq \varphi(x)$ and $x^{(n)} \coloneqq \gamma_n(\varphi(x)) \in
		\Gamma^n(M)$, following notation of \cref{defn:dividedPowers}.
		Then the $A$-module $\Gamma^n(M)$ is generated by
		\begin{equation*}
		\left\{ x_1^{(q_1)} \cdots x_k^{(q_k)} \ \middle|\ 
		q_1 + \cdots + q_k = n \right\}
		.\end{equation*}
		Moreover, if $\{ x_i \}_{i \in I}$ is a basis
		for $M$, then $\{ x^{(n)}_i \}_{I \in I}$
		is a basis for $\Gamma^n(M)$,
		for all $n \geq 1$.
\end{enumerate}
	To make notation cleaner, when the base $A$ is clear, we
	will write $\Gamma(M)$ for $\Gamma_A(M)$.
\end{thm}


\noindent
The point of introducing all of these definitions is to allow one to define the following
inverse maps.
\begin{defn}[]
	Let $\left(A, I, \gamma\right)$ be a nilpotent divided powers ring.
	Then we can define two maps
	\begin{equation*}
	\begin{tikzcd}[row sep = 0ex
		,/tikz/column 1/.append style={anchor=base east}
		,/tikz/column 2/.append style={anchor=base west}]
		\exp\colon I \arrow[r, "", rightarrow] &
		\left( 1 + I \right)^*\\
		\log\colon \left( 1 + I \right)^* \arrow[r, "", rightarrow] &
		I
	,\end{tikzcd}
	\end{equation*} 
	given by 
	$\exp (x) \coloneqq \sum_{n\geq 0} \gamma_n(x)$
	and $\log (1+x) \coloneqq \sum_{n\geq 1} 
	(-1)^{n-1} \left( n-1 \right)! \gamma_n(x)$.
	Then, as outlined in \cite[Chapter III, \S1.6]{Messing},
	one checks that these maps are inverses to each other
	by reducing to the universal case $\widehat{\Gamma_{\mathbb{Z}}(\mathbb{Z})}$.
\end{defn}


\begin{rem}[]
	The above constructions can all be globalized to the case of
	$\mathcal{O}_S$ algebras and modules, where $S$ denotes a scheme.
	In particular one needs to replace $A$ by the above mentioned scheme $S$,
	$I$ by a quasi-coherent sheaf of ideals $\mathscr{I}$ of $\mathcal{O}_S$ and
	$M$ by a quasi-coherent $\mathcal{O}_S$-module.
	Then a system of divided powers on $\mathscr{I}$ is the data
	of, for all $U \subset S$ open, a system of divided powers of
	$\Gamma(U,\mathscr{I})$ in which divided power morphisms and restriction
	maps commute.
	Then we will call one such triple $\left(S, \mathscr{I}, \gamma\right)$
	a {\em scheme with divided powers}.

	Moreover one generalizes morphism of divided powers in the following way:
	let $\left(S, \mathscr{I}, \gamma\right)$ and $\left(S', \mathscr{I}', \gamma'\right)$
	be schemes with divided powers.
	A {\em morphism of divided powers}
	\begin{equation*}
	f\colon \left(S, \mathscr{I}, \gamma\right) \to 
	\left(S', \mathscr{I}', \gamma'\right)
	\end{equation*}
	is a morphism of schemes
	$f\colon S \to S'$ such that
	$f^{-1}(\mathscr{I}') \mathcal{O}_S \subset \mathscr{I}$
	and, for all $U' \subset S'$ open,
	\begin{equation*}
	\begin{tikzcd}[row sep = 0ex
		,/tikz/column 1/.append style={anchor=base east}
		,/tikz/column 2/.append style={anchor=base west}]
		f^\#(U')\colon 
		\left(\mathcal{O}_{S'}(U'), \mathscr{I}'(U'), \gamma'\right)
		\arrow[r, "", rightarrow] &
		\left(\mathcal{O}_{ S }(f^{-1}U'), \mathscr{I}(f^{-1}U'), \gamma\right)
	\end{tikzcd}
	\end{equation*} 
	is a morphism of divided powers rings.

	Moreover the divided powers algebra $\Gamma(M)$ is defined
	as the sheaf associated to the presheaf 
	$U \mapsto \Gamma_{\mathcal{O}_S(U)}(M(U))$.
	Finally the divided powers on an ideal $\mathscr{I} \subset \mathcal{O}_S$
	are said to be {\em nilpotent} iff,
	locally on $S$, they satisfy conditions in \cref{defn:NilpotentDividedPowers}.
\end{rem}


\subsection{}
The notation of this section will follow that of \cite[Capther III]{Messing}.
This means that it might not be consistent with our previous exposition.
\begin{defn}[Quasi-coherent co-algebra]
	Let $S$ be a scheme.
	We say that $\mathscr{U}$ is a quasi-coherent $\mathcal{O}_{ S }$ {\em co-algebra}
	iff it is given with an $\mathcal{O}_{ S }$-algebra structure
	and morphisms of $\mathcal{O}_S$-algebras
	$\Delta\colon U \to U \otimes_{\mathcal{O}_{ S }} U$
	and $\eta\colon U \to \mathcal{O}_{ S }$
	satisfying the properties of Hopf algebra morphisms,
	as defined in \cref{defn:HopfAlgebra}.
\end{defn}


\begin{defn}[Cospec]
	Let $S$ be a scheme and $U$ an $\mathcal{O}_{ S }$ co-algebra.
	We define the functor 
	\begin{equation*}
	\begin{tikzcd}[row sep = 0ex
		,/tikz/column 1/.append style={anchor=base east}
		,/tikz/column 2/.append style={anchor=base west}]
		\mathrm{Cospec}(U)\colon 
		\mathsf{Sch}_{ S }\arrow[r, "", rightarrow] &
		\mathsf{Sets} \\
		S' \arrow[r, "", mapsto] & 
		\left\{ y \in \Gamma(S', U_{S'}) \ \middle|\ 
		\eta(y) = 1, \Delta(y) = y \otimes y\right\}
	.\end{tikzcd}
	\end{equation*} 
\end{defn}


\begin{rem}[{\cite[Chapter 3, \S2.1]{Messing}}]
	The functor $\mathrm{Cospec}(U)$ is a sheaf for the fpqc topology
	for all $\mathcal{O}_{ S }$ co-algebra $U$.
\end{rem}
\end{document}
