\section*{Introduction}
In his renown paper, \cite{TatePC}, Tate proved a very explicit Hodge-like decomposition
of the Tate module of a Barsotti-Tate group, when tensored with $\mathbb{C}_p$.
In particular, when working over abelian schemes, this granted a decomposition of the
(tk) etale cohomology with coefficients in $\mathbb{Q}_p$.
It was this paper which started the development of the field that became $p$-adic Hodge theory,
in the push from Tate to find analogous decompositions for the (tk)etale cohomology
with values in $\mathbb{C}_p$ for schemes admitting a proper and smooth model over
the ring of integers of some local field $K$.
This result was achieved by Fontaine using its so called {\em period rings},
which actually allowed him to go even further and obtain finer results in the
study of cohomology theory.
In particular, when dealing with a variety $X$ over $K$, admitting
a proper smooth model over $O_K$, tensoring with the period ring $B_{\mathrm{cris}}$
grants an isomorphism between de Rham and crystalline cohomology.
In the case of abelian varieties (tk)etale cohomology is just the dual
of the Tate module of its associated Barsotti-Tate group,
whereas the (tk)Dieudonne module represents crystalline cohomology.
The aim of this text is to prove a crystalline comparison theorem which
grants an isomorphism between the just mentioned objects, after tensoring
with the period ring $B_{\mathrm{cris}}$.
To achieve this goal we will introduce the basic language of group schemes, which
will allow the definition of $p$-divisible groups.
The notion of sheaves on sites will be also introduced to give some
more tools to work with Barsotti-Tate groups and Dieudonne modules.
We will then introduce the theory of Universal extensions which allows the introduction
of crystals associated to Barsotti-Tate group, a generalization of the classical
idea of (tk)Dieudonne modules.
These, together with the theory of deformation of Grothendieck-Messing,
will then be used to introduce the theory of classification
of $p$-divisible groups over $O_K$ of Kisin-Breuil.
Then, after spending some time to introduce the most relevant period rings
for the aim of this work, we will use the above-mentioned results
to construct and study the desired comparison isomorphism.



\section*{Notation and conventions}
Each time we will use the letter $p$ it is going to denote a prime number.
For convenience sake we can fix it here once and for all.
With this in mind we will denote by $\mathbb{F}_{p}$ the field with $p$ elements,
by $\mathbb{Z}$ the ring of integers and by $\mathbb{Q}_p$ and $\mathbb{Z}_{p}$
respectively the field and ring of $p$-adic numbers.
Then, for an extension $K/\mathbb{Q}_p$, we will denote by $O_K$ the ring of
integers of $K$ and by $K_0$ the maximal unramified subextension of $K/\mathbb{Q}_p$.

With regards to algebraic geometry: for a morphism of schemes $f\colon X \to Y$,
we will denote by $f^{\#}\colon \mathcal{O}_{ Y } \to f_*\mathcal{O}_{ X }$ the
associated morphism between structure sheaves and, given $y \coloneqq f(x)$,
by $f_y^{\#}\colon \mathcal{O}_{ Y,y } \to \mathcal{O}_{ X,x }$ the induced morphism
at the level of stalks.
Finally we denote by $\mathfrak{m}_x$ the maximal ideal of the local ring
$\mathcal{O}_{ X,x }$ and by $\kappa(x) \coloneqq \mathcal{O}_{ X,x }/\mathfrak{m}_x$
the residue field at $x$.

With respect to category theory:
we will not worry about universes. More precisely we will
tacitly assume a universe $\mathbb{U}$ has been chosen
and we restrict to categories whose objects lie in $\mathbb{U}$.
We will denote categories using a sans serif font.
In particular we will use the following notations:
$\mathsf{Sch}_{S}$ for the category of schemes over $S$,
$\mathsf{Gp}$ for that of groups,
$\mathsf{Ab}$ for that of abelian groups,
$\mathsf{Top}$ for that of topological spaces,
and $\mathsf{Sets}$ for that of sets.
Moreover, by {\em ring} or {\em algebra}, we will mean one which is commutative and with unity.
Finally we will often use the shorter $X \in \mathsf{C}$ to mean that $X$ is an object
of the category $\mathsf{C}$, i.e. that $X \in \mathrm{Ob} \left(\mathsf{C}\right)$.
