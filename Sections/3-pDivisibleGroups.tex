\documentclass[../Main]{subfiles}
\begin{document}
\section{\texorpdfstring{$p$}{p}-divisible groups}
In all that follows $p$ will be a fixed prime number.
Moreover $R$ denotes a commutative ring.

\begin{defn}[$p$-divisible group]
	A {\em $p$-divisible group} over $R$ of height $h \in \mathbb{N}_+$ is an inductive system
	\begin{equation*}
	G \coloneqq \left(G_\nu, i_\nu\right)_{\nu \in \mathbb{N}}
	,\end{equation*} 
	satisfying:
	\begin{enumerate}
		\item for each $\nu \in \mathbb{N}$, $G_\nu$ is a finite, commutative group scheme over $R$
			of order $p^{\nu h}$,
		\item for each $\nu \in \mathbb{N}$, there is an exact sequence
			\begin{equation*}
			\begin{tikzcd}
				0 \arrow[r, "", rightarrow] &
				G_\nu \arrow[r, "i_\nu", rightarrow] &
				G_{\nu + 1} \arrow[r, "p^\nu", rightarrow] &
				G_{\nu + 1} 
			,\end{tikzcd}
			\end{equation*}
			where the second map is the multiplication by $p^\nu$ in $G_{\nu + 1}$,
			hence the first is a closed immersion.
	\end{enumerate}
\end{defn}

\begin{rem}[]
	In the case of ordinary abelian groups this would give rise just to 
	$G_\nu = \left( \mathbb{Z}/p^\nu\mathbb{Z} \right)^h$, hence to the $p$-divisible group
	\begin{equation*}
		G = \varinjlim G_\nu = \left( \mathbb{Q}_p / \mathbb{Z}_p \right)^h
	.\end{equation*} 
\end{rem}


\begin{defn}[Formal Lie group]
	Let $R$ be a complete, noetherian, local ring,
	with residue field $k$ of characteristic $p > 0$.
	Let $\mathscr{A} \coloneqq R [\![ X_1, \ldots, X_{ n } ]\!]$
	be the ring of formal power series in $n$ variables,
	and $\mathscr{A} \widehat{\otimes}_R \mathscr{A}$ the ring of formal power series 
	in $2n$ variables.
	An {\em $n$-dimensional formal Lie group $\Gamma$ over $R$} is 
	given by $\Gamma \coloneqq \mathrm{Spec}\, \mathscr{A}$,
	(tk: are you sure? The more I read, the more I am {\em un}sure...)
	where the group law is defined by a homomorphism of $R$-algebras 
	\begin{equation*}
	\begin{tikzcd}[row sep = 0ex
		,/tikz/column 1/.append style={anchor=base east}
		,/tikz/column 2/.append style={anchor=base west}]
		f\colon \mathscr{A} \arrow[r, "", rightarrow] &
		\mathscr{A} \widehat{\otimes}_R \mathscr{A}
	,\end{tikzcd}
	\end{equation*} 
	satisfying the following conditions:
	\begin{enumerate}
		\item {\em $\varepsilon$ axiom:} $X = f(X,0) = f(0,X)$,
		\item {\em coassociativity:} $f(X, f(Y,Z)) = f(f(X,Y), Z)$,
		\item {\em commutativity:} $f(X,Y) = f(Y,X)$.
	\end{enumerate}
	Notice that any such morphism $f$ is just the data of 
	$\left( f_i(Y,Z) \right)_{i=1}^n$, power series in $2n$ variables, where
	$f_i$ is the image of $X_i$ via $f$.
	Then one, at the level of $R$-algebras, introduces the notation
	\begin{equation*}
		X \ast Y \coloneqq f(X,Y)
	.\end{equation*} 
\end{defn}

\begin{rem}[]
	These axioms are enough to grant the existance of the inverse for any element of $\Gamma$,
	hence they suffice to give $\Gamma$ a group scheme structure.
\end{rem}


\begin{ex}[]
	We define the map multiplication by $p$ on $\Gamma$ as the map $p\colon \Gamma \to \Gamma$
	associated to 
	\begin{equation*}
	\begin{tikzcd}[row sep = 0ex
		,/tikz/column 1/.append style={anchor=base east}
		,/tikz/column 2/.append style={anchor=base west}]
		\psi\colon \mathscr{A} \arrow[r, "", rightarrow] &
		\mathscr{A} \widehat{\otimes}_R \mathscr{A} \\
		X \arrow[r, "", mapsto] & 
		X \ast \ldots \ast X
	\quad \text{($p$ times)}
	.\end{tikzcd}
	\end{equation*} 
	A formal Lie group $\Gamma$ is said to be {\em divisible} iff the map $p$ is 
	an isogeny, i.e. it is surjective and has finite kernel.
	In such case one can define a $p$-divisible group of height $h$
	over $R$, starting from $\Gamma$, by:
	\begin{equation*}
		\Gamma(p) \coloneqq \left(\Gamma_{p^\nu}, i_{p^\nu}\right)_\nu
	.\end{equation*} 
	In the above $\Gamma_{p^\nu}$ is the kernel of the multiplication by $p^\nu$ 
	in $\Gamma$.
	By questions of connectedness (tk: see short exact sequence of connected
	and etale groups) one sees that the order of $\Gamma_{p^\nu}$
	is a power of $p$, that $\Gamma_{p^\nu}$ is connected.
	hence that $\Gamma(p)$ is a connected $p$-divisible group.

	tk: complete it.
\end{ex}

\begin{prop}[]
	Let $R$ be a complete noetherian local ring whose residue field $k$
	is of characteristic $p > 0$.
	Then $\Gamma \mapsto \Gamma(p)$ is an equivalence of categories
	between the category of divisible commutative formal Lie groups over $R$
	and the category of connected $p$-divisible groups over $R$.
\end{prop}

\begin{defn}[Dimension of a $p$-divisible group]
	Let $G \coloneqq \left(G_{\nu}, i_\nu\right)_{\nu \in \mathbb{N}}$ be a $p$-divisible group
	over $R$ as before.
	The connected components $G^0_\nu$ determine a connected $p$-divisible
	group $G^0$.
	Moreover, from the short exact sequence
	\begin{equation*}
	\begin{tikzcd}
		0 \arrow[r, "", rightarrow] &
		G^0_\nu \arrow[r, "", rightarrow] &
		G_\nu \arrow[r, "", rightarrow] &
		G^{et}_\nu \arrow[r, "", rightarrow] &
		0
	\end{tikzcd}
	\end{equation*}
	one gets the exact sequence
	\begin{equation*}
	\begin{tikzcd}
		0 \arrow[r, "", rightarrow] &
		G^0 \arrow[r, "", rightarrow] &
		G \arrow[r, "", rightarrow] &
		G^{et} \arrow[r, "", rightarrow] &
		0
	,\end{tikzcd}
	\end{equation*}
	where $G^{et}$ is an e\'tale $p$-divisible group.
	One then defines the dimension of $G$ to be the dimension 
	(i.e. the number of variables of $\mathscr{A}$)
	of the formal Lie group corresponding, as of proposition (tk: reference to the above one),
	to $G^0$.
\end{defn}


\end{document}
