\section{Comparison morphism}
We introduce, for this section, the following notation:
$K$ will denote a complete discrete valuation field, 
with perfect residue field $k$ of characteristic $p$.
We will denote by $W \coloneqq W(k)$ the ring of Witt vectors with coefficients in $k$
and by $K_0 \coloneqq W[1/p]$ its field of fractions.
We will denote by $\sigma$ the absolute Frobenius acting
on $k, W$ and $K_0$.
We fix $e \coloneqq [ K : K_0 ]$ the absolute ramification index of $K$,
we denote by $\overline{K}$ a fixed separable closure of $K$,
and by $\mathbb{C}_K \coloneqq \widehat{\overline{K}}$ its completion.
Finally we will denote by $\mathscr{G}_K \coloneqq \mathrm{Gal}\left( \overline{K} / K \right)$ 
the absolute Galois group of $K$ and notice that its action on $\overline{K}$
extends to $\mathbb{C}_K$ by continuity.



\subsection{Galois representations}
\begin{defn}[$p$-adic representation]
	A {\em $p$-adic representation} of a profinite group $\Gamma$ is a
	representation $\rho\colon \Gamma \to \mathrm{Aut}_{\mathbb{Q}_p}(V)$
	of $\Gamma$ on a finite-dimensional $\mathbb{Q}_p$ vector space
	$V$, where $\rho$ is continuous.
	Here the topology on $\mathrm{Aut}_{\mathbb{Q}_p}(V)$ is
	that of $\mathrm{GL}_n(\mathbb{Q}_p)$, which is well defined,
	independently of the chosen basis.

	A {\em morphism of $p$-adic representations} $V_1, V_2$ of $\Gamma$
	is a $\Gamma$-equivariant linear map $f\colon V_1 \to V_2$, 
	i.e. a linear map that commutes with the action of $\Gamma$.
	More explicitly, denoting $\rho_i(\gamma)$ simply by $\gamma$ for
	an element $\gamma \in \Gamma$, a $\Gamma$-equivariant
	map satisfies $\gamma(f(v)) = f(\gamma(v))$
	for all $v \in V_1$ and all $\gamma \in \Gamma$.

	We denote the category of $p$-adic representations of $\Gamma$,
	whose objects and morphism have just been described,
	by $\mathsf{Rep}_{\mathbb{Q}_p}(\Gamma)$.
\end{defn}


\begin{rem}[]
	In general the above definition is used for representations of Galois groups.
	In particular $\mathscr{G}_K$ is profinite and one studies
	$p$-adic representations of $\Gamma = \mathscr{G}_K$.
\end{rem}


%\begin{rem}[]
%	tk: maybe one can add the example of Tate module of an elliptic curve here.
%	Define a Tate module before though.
%	See page 131 of my notes.
%\end{rem}


\begin{defn}[$(F,G)$-regular algebra]
	Let $F$ be a field and $G$ be a group.
	Let $B$ be an $F$-algebra domain
	equipped with an action of $G$ via automorphisms of $F$-algebras.
	Denote by $C \coloneqq \mathrm{Frac}(B)$ and by $E \coloneqq B^{G}$.
	Notice that $G$ acts on $C$ in a natural way.
	We say that $B$ is {\em $(F,G)$-regular} iff
\begin{enumerate}
	\item $B^{G} = C^{G}$ and
	\item if $b \in B$ generates a vector space $F \cdot b$
		stable under the action of $G$, then $b \in B^{\cross}$.
\end{enumerate}
\end{defn}


\begin{rem}[]
	Notice that for any $(F,G)$-regular algebra $E/F$ is a field extension.
	Moreover if $B$ is already a field, than it clearly is $(F,G)$-regular.
	Finally we will always be concerned with $G = \mathscr{G}_K$ and $F = \mathbb{Q}_p$,
	so we will fix them and assume that $B$ is a $(\mathbb{Q}_p, \mathscr{G}_K)$-regular
	algebra.
	Moreover in the following we will simply write 
	$G$-regular to mean $(\mathbb{Q}_p, G)$-regular.
\end{rem}


\begin{defn}[]
	One can define the functor
	\begin{equation*}
	\begin{tikzcd}[row sep = 0ex
		,/tikz/column 1/.append style={anchor=base east}
		,/tikz/column 2/.append style={anchor=base west}]
		\mathbf{D}_B\colon \mathsf{Rep}_{\mathbb{Q}_p}(\mathscr{G}_K) \arrow[r, "", rightarrow] &
		\mathsf{Vect}(E) \\
		V \arrow[r, "", mapsto] & 
		\mathbf{D}_B(V) \coloneqq \left( B \otimes_{\mathbb{Q}_p} V \right)^{\mathscr{G}_K}
	,\end{tikzcd}
	\end{equation*} 
	where we denoted by $\mathsf{Vect}(E)$ the category of $E$-vector spaces.
	Moreover we can define a natural $B$-linear, $\mathscr{G}_K$-equivariant map
	\begin{equation*}
	\begin{tikzcd}[row sep = 0ex
		,/tikz/column 1/.append style={anchor=base east}
		,/tikz/column 2/.append style={anchor=base west}]
		\alpha_B(V)\colon 
		B \otimes_E \mathbf{D}_B(V) \arrow[r, "", rightarrow] &
		B \otimes_{\mathbb{Q}_p} V \\
		b \otimes d
		\arrow[r, "", mapsto] & 
		bd
	.\end{tikzcd}
	\end{equation*}
\end{defn}


\begin{prop}[{\cite[Theorem 5.2.1]{Brinon}}]
	Fix $V \in \mathsf{Rep}_{\mathbb{Q}_p}(\mathscr{G}_K)$.
	Then the map $\alpha_B(V)$ is always injective and
	$\dim_E \mathbf{D}_B(V) \leq \dim_E V$.
	Moreover there is equality of dimensions iff $\alpha_B(V)$ is an isomorphism.
\end{prop}


\begin{defn}[Admissible representations]
	We say that $V \in \mathsf{Rep}_{\mathbb{Q}_p}(\mathscr{G}_K)$ is a
	{\em $B$-admissible representation} iff 
	$\dim_E \mathbf{D}_B(V) = \dim_E V$.
	We denote by $\mathsf{Rep}_{\mathbb{Q}_p}^B(\mathscr{G}_K) \subset
	\mathsf{Rep}_{\mathbb{Q}_p}(\mathscr{G}_K)$
	the full subcategory of $B$-admissible representations.
\end{defn}


\begin{prop}[{\cite[Theorem 5.2.1]{Brinon}}]
	The functor
	\begin{equation*}
	\begin{tikzcd}[row sep = 0ex
		,/tikz/column 1/.append style={anchor=base east}
		,/tikz/column 2/.append style={anchor=base west}]
		\mathbf{D}_B\colon \mathsf{Rep}_{\mathbb{Q}_p}^B(\mathscr{G}_K) \arrow[r, "", rightarrow] &
		\mathsf{Vect}_f(E) \\
		V \arrow[r, "", mapsto] & 
		\mathbf{D}_B(V) \coloneqq \left( B \otimes_{\mathbb{Q}_p} V \right)^{\mathscr{G}_K}
	,\end{tikzcd}
	\end{equation*} 
	where we denoted by $\mathsf{Vect}_f(E)$ the category of finite dimensional $E$-vector spaces,
	is exact and faithful.
	Moreover any subrepresentation and quotient of a $B$-admissible
	representation is $B$-admissible.
\end{prop}



\subsection{Period rings}
\begin{defn}[]
	Let $A$ be an $\mathbb{F}_p$-algebra.
	We can associate it the perfect $\mathbb{F}_p$ algebra
	\begin{equation*}
		R(A) \coloneqq \varprojlim_{x \mapsto x^p} A =
		\left\{ \mathbf{x} = \left( \mathbf{x}_0, \mathbf{x}_1, \ldots \right) 
			\in \prod_{n \in \mathbb{N}} A
		\ \middle|\ \mathbf{x}_{n+1}^p = \mathbf{x}_n \text{ for all } n \in \mathbb{N} \right\}
	\end{equation*}
	endowed with the product ring structure.
\end{defn}


\begin{rem}[]\leavevmode\vspace{-.2\baselineskip}
\begin{enumerate}
\item The above $\mathbb{F}_p$-algebra is perfect
	since the $p$-th power map is clearly surjective by definition.
	Moreover it is injective since any element $\mathbf{x} = \left( \mathbf{x}_n \right)$
	satisfying $\mathbf{x}^p = 0$ has
	$\mathbf{x}_{n-1} = \mathbf{x}_n^p = 0$ for all $n \geq 1$.

\item We have a canonical morphism
	\begin{equation*}
	\begin{tikzcd}[row sep = 0ex
		,/tikz/column 1/.append style={anchor=base east}
		,/tikz/column 2/.append style={anchor=base west}]
		R(A) \arrow[r, "", rightarrow] &
		A \\
		\left( \mathbf{x}_n \right)_{n \in \mathbb{N}} \arrow[r, "", mapsto] & 
		\mathbf{x}_0
	.\end{tikzcd}
	\end{equation*} 
	Moreover any morphism from a perfect $\mathbb{F}_p$-algebra
	to $A$ factors through the above projection.

\item If $A$ is already perfect, then $R(A) \simeq A$.
	In particular the inverse map of the above projection is given by
	$a \mapsto \left( a^{1/p^n} \right)$.
	In particular, let's consider $\overline{k}$ a fixed separable closure of $k$.
	Since $\overline{k}$ is already perfect, we obtain $R(\overline{k}) \simeq \overline{k}$.

\item It can be shown that, given $F$ a field of characteristic $p$, then $R(F)$ is the largest
	perfect subfield of $F$.
\end{enumerate}
\end{rem}


\begin{ntt}[]
	We introduce the following ring
	\begin{equation*}
		R \coloneqq R(O_{\mathbb{C}_K}/pO_{\mathbb{C}_K}) =
		R(O_{\overline{K}}/pO_{\overline{K}})
	.\end{equation*}
	It is a perfect $\mathbb{F}_p$-algebra and also canonically an
	algebra over $\overline{k}$, since $O_{\mathbb{C}_K}$ is.
\end{ntt}


\begin{prop}[{\cite[Proposition 4.3.1]{Brinon}}]
	Let $O$ be a $p$-adically separated and complete ring, and $\mathfrak{a} \triangleleft O$
	an ideal of $O$ containing $p$ and such that $\mathfrak{a}^N \subset pO$ for some $N \in \mathbb{N}$
	(i.e. the $\mathfrak{a}$-adic and $p$-adic topologies coincide).
	Then we have a map
	\begin{equation*}
	\begin{tikzcd}[row sep = 0ex
		,/tikz/column 1/.append style={anchor=base east}
		,/tikz/column 2/.append style={anchor=base west}]
		R(O/\mathfrak{a}) \arrow[r, "", rightarrow] &
		\varprojlim_{x \mapsto x^p} O \\
		\left( \mathbf{x}_n \right)_{n \in \mathbb{N}} \arrow[r, "", mapsto] & 
		\left( \mathbf{x}^{(n)} \right)_{n \in \mathbb{N}}
	,\end{tikzcd}
	\end{equation*} 
	where we define $\mathbf{x}^{(n)} \coloneqq \lim_{m \to \infty} \widehat{\mathbf{x}_{n+m}}^{p^m}$,
	in which $\widehat{\mathbf{x}_m}$ is any lift of $\mathbf{x}_m$ to $O$.
	This map does not depend on the choice of lift, it is bijective
	and its inverse is given by
	\begin{equation*}
	\begin{tikzcd}[row sep = 0ex
		,/tikz/column 1/.append style={anchor=base east}
		,/tikz/column 2/.append style={anchor=base west}]
		\varprojlim_{n \mapsto x^p} O \arrow[r, "", rightarrow] &
		R(O/\mathfrak{a}) \\
		\left( \mathbf{x}^{(n)} \right)_{n \in \mathbb{N}} \arrow[r, "", mapsto] & 
		\left( \mathbf{x}^{(n)} \mathrm{mod}\, \mathfrak{a} \right)_{n \in \mathbb{N}}
	.\end{tikzcd}
	\end{equation*} 
	Moreover $R(O/pO) \simeq R(O/\mathfrak{a})$ and this common ring
	is a domain as soon as $O$ is.
\end{prop}


\begin{rem}[]\label{tiltingOperations}
	If we endow $\varprojlim_{x \mapsto x^p} O$ with the ring structure
	given, for any $\mathbf{x} \coloneqq \left( \mathbf{x}^{(n)} \right)$ and 
	$\mathbf{y} \coloneqq \left( \mathbf{y}^{(n)} \right)$, 
	by
	\begin{equation*}
		\left( \mathbf{xy} \right)^{(n)} \coloneqq \mathbf{x}^{(n)} \mathbf{y}^{(n)}
		\qquad \text{ and } \qquad
		\left( \mathbf{x} + \mathbf{y} \right)^{(n)} \coloneqq
		\lim_{m \to \infty} \big( \mathbf{x}^{(n+m)} +
		\mathbf{y}^{(n+m)}\big)^{p^m}
	,\end{equation*}
	then the above bijection is an isomorphism of rings.
\end{rem}


\begin{ntt}[]\label{not:tiltingElts}
	In view of the above isomorphism we might see an element $\mathbf{x} \in R$
	either as an element 
	$\left( \mathbf{x}_n \right) \in \varprojlim_{x \mapsto x^p} O_{\mathbb{C}_K}/ (p)$
	or as an element
	$\left( \mathbf{x}^{(n)} \right) \in \varprojlim_{x \mapsto x^p} O_{\mathbb{C}_K}$.
	We will use the above notation accordingly.
\end{ntt}


\begin{rem}[]\label{rem:GKActionR}
	Let's now notice that $\mathscr{G}_K \coloneqq \mathrm{Gal}\left( \overline{K} / K \right)$
	acts naturally on $O_{\mathbb{C}_K}$, since it acts via isometries on
	$\overline{K}$.
	Acting via morphisms of rings, which commute with $x \mapsto x^p$,
	its action can naturally be extended to
	$R = \varprojlim_{x \mapsto x^p} O_{\mathbb{C}_K}$.
\end{rem}


\begin{lem}[{\cite[Lemma 4.3.3]{Brinon}}]
	Denote by $\left| \ \cdot \ \right|_p$ the absolute value on $\mathbb{C}_K$
	normalized by $\left| p \right|_p = 1/p$.
	The map
	\begin{equation*}
	\begin{tikzcd}[row sep = 0ex
		,/tikz/column 1/.append style={anchor=base east}
		,/tikz/column 2/.append style={anchor=base west}]
		\left| \ \cdot \ \right|_R\colon R \arrow[r, "", rightarrow] &
		p^{\mathbb{Q}} \cup \left\{ 0 \right\} \\
		\left( \mathbf{x}^{(n)} \right)_{n \in \mathbb{N}} \arrow[r, "", mapsto] & 
		\left| \mathbf{x}^{(0)} \right|_p
	\end{tikzcd}
	\end{equation*} 
	is a $\mathscr{G}_K$-equivariant absolute value on $R$ that makes
	$R$ the valuation ring for the unique valuation
	$\nu_R$ on $\mathrm{Frac}\, R$ extending
	$-\log_p \left| \ \cdot \ \right|_R$ on $R$
	and having value group $\mathbb{Q}$.
	Moreover $R$ is $\nu_R$-adically separated and complete
	and the subfield $\overline{k}$ of $R$ maps
	isomorphically onto the residue field of $R$.
\end{lem} 


\begin{ex}[]\leavevmode\vspace{-.2\baselineskip}\label{ExampleEltsTilt}
\begin{enumerate}
\item Fix $\big( p^{1/p^n} \big)_{n \in \mathbb{N}}$ 
	a compatible family of $p^n$-th roots of $p$
	in $O_{\mathbb{C}_K}$.
	Denote by $\mathbf{p}$ the element of $R$ given by
	\begin{equation*}
		\mathbf{p} \coloneqq \big( p^{(n)} \big)_{n \in \mathbb{N}} =
		\big( p, p^{1/p}, p^{1/p^2}, \ldots \big) \in R
	.\end{equation*}
	Its valuation is easily computed by
	$\nu_R(\mathbf{p}) = \nu_p(p^{(0)}) = \nu_p(p) = 1$.

\item Fix a compatible family of primitive $p^n$-th roots of unity
	$\left( \zeta_{p^n} \right)_{n \in \mathbb{N}}$ in $O_{\mathbb{C}_K}$.
	We denote by $\boldsymbol\varepsilon$ the special element of $R$ given by
	\begin{equation*}
		\boldsymbol\varepsilon \coloneqq \big( \varepsilon^{(n)} \big)_{n \in \mathbb{N}} =
		\left( 1, \zeta_p, \zeta_{p^2}, \ldots \right) \in R
	.\end{equation*}
	The element $\boldsymbol\varepsilon$ depends on the chosen
	compatible family.
	Moreover any two such $\boldsymbol\varepsilon$
	are $\mathbb{Z}_{p}^{\cross}$-powers of each other.
	Moreover we have
	$\nu_R \left( \boldsymbol\varepsilon - 1 \right) = p/ (p - 1)$.
	Let's show this for $p > 2$: by definition we have
	$\nu_R(\boldsymbol\varepsilon - 1) = \nu_p \left( (\boldsymbol\varepsilon -1 )^{(0)} \right)$.
	By \cref{tiltingOperations} we have
	\begin{equation*}
		\left( \boldsymbol\varepsilon -1 \right)^{(0)} =
		\lim_{n \to \infty} \left( \zeta_{p^n} + (-1)^{(n)} \right)^{p^n}
	.\end{equation*}
	Let's notice that $\left( -1 \right)^{(n)} = -1$ for all $n$
	and that $\zeta_{p^n} - 1$ is a root of $\Phi_{p^n}(1+X)$,
	where $\Phi_m$ denotes the $m$-th cyclotomic polynomial.
	In particular $\Phi_{p^n}(1+X)$ is Eisenstein of degree
	$p^{n-1}(p-1)$.
	Then
	\begin{equation*}
		\nu_R(\boldsymbol\varepsilon - 1) =
		\lim_{n \to \infty} \frac{ p^n }{ p^{n-1}(p-1) } =
		\frac{ p }{ p-1 }
	.\end{equation*}
	Finally we recall that $\mathscr{G}_K$ acts on $\zeta_{p^n}$ via
	the cyclotomic character, which is defined by
	$g(\zeta_{p^n}) = \zeta_{p^n}^{\chi(g)}$ for any $g \in \mathscr{G}_K$.
	As a consequence, since the induced action is component-wise,
	$\mathscr{G}_K$ acts also on $\boldsymbol\varepsilon$ via the cyclotomic character,
	i.e. $g(\boldsymbol\varepsilon) = \boldsymbol\varepsilon^{\chi(g)}$
	for all $g \in \mathscr{G}_K$.
\end{enumerate}
\end{ex}


\begin{thm}[{\cite[Theorem 4.3.5]{Brinon}}]
	The field $\mathrm{Frac}\, R = R[1/\mathbf{p}]$ is algebraically closed.
\end{thm}


\begin{rem}[]
	There is a natural family of ring homomorphisms
	\begin{equation*}
	\begin{tikzcd}[row sep = 0ex
		,/tikz/column 1/.append style={anchor=base east}
		,/tikz/column 2/.append style={anchor=base west}]
		\theta_m\colon R \arrow[r, "", rightarrow] &
		O_{\mathbb{C}_K} \\
		\left( \mathbf{x}_m \right)_{m \in \mathbb{N}} \arrow[r, "", mapsto] & 
		\mathbf{x}_n
	.\end{tikzcd}
	\end{equation*} 
	Let's give $R$ a $\overline{k}$-algebra structure
	via the $k$-embedding
	\begin{equation*}
	\begin{tikzcd}[row sep = 0ex
		,/tikz/column 1/.append style={anchor=base east}
		,/tikz/column 2/.append style={anchor=base west}]
		\overline{k} = R(\overline{k}) \arrow[r, "", rightarrow] &
		R(O_{\overline{K}}/pO_{\overline{K}}) =R\\
		c \arrow[r, "", mapsto] & 
		\left( j(c), j(c^{1/p}), j(c^{1/p^2}), \ldots \right)
	,\end{tikzcd}
	\end{equation*} 
	where $j\colon \overline{k} \to O_{\overline{K}}/ (p)$ is the canonical
	section to the reduction map $O_{\overline{K}}/ (p) \twoheadrightarrow \overline{k}$.
	Then $\theta_0$ is a morphism of $\overline{k}$-algebras.
	We wish to lift it to a ring map $W(R) \to O_{\mathbb{C}_K}$,
	but we cannot use universal property of the Witt vector construction
	since $O_{\mathbb{C}_K}/ (p)$ is not perfect (in particular the Frobenius
	is not injective).
\end{rem}


\begin{defn}[]
	We define, set theoretically, the map
	\begin{equation*}
	\begin{tikzcd}[row sep = 0ex
		,/tikz/column 1/.append style={anchor=base east}
		,/tikz/column 2/.append style={anchor=base west}]
		\theta\colon W(R) \arrow[r, "", rightarrow] &
		O_{\mathbb{C}_K} \\
		\sum_{n \in \mathbb{N} }^{  } [\mathbf{c}_n] p^n \arrow[r, "", mapsto] & 
		\sum_{n \in \mathbb{N} }^{  } \mathbf{c}_n^{(0)} p^n
	,\end{tikzcd}
	\end{equation*} 
	where \cref{TeichmullerExpansionWitt} allows us to write any element of $W(R)$
	in a unique Teichmüller expansion and $\mathbf{c}_n^{(0)}$ is defined as in \cref{not:tiltingElts}.
\end{defn}


\begin{rem}[]
	In \cref{TeichmullerExpansionWitt} we explicitly computed the Teichmüller
	expansion of $\left( \mathbf{r}_n \right)_{n \in \mathbb{N}}$
	to be $\sum_{n \in \mathbb{N} }^{  } [\mathbf{r}_n^{p^{-n}}] p^n$.
	Moreover, by compatibility of the elements in $\varprojlim_{x \mapsto x^p} O_{\mathbb{C}_K}$
	and multiplicativity of $\mathbf{r} \mapsto \mathbf{r}^{(n)}$
	we have that $\big( \mathbf{r}^{p^{-n}} \big)^{(0)} = \big( (\mathbf{r}^{p^{-n}})^{(n)} \big)^{p^n} =
	\mathbf{r}^{(n)}$.
	Hence we can compute $\theta$ also via
	\begin{equation*}
	\begin{tikzcd}
		\theta\colon 
		\left( \mathbf{r}_0, \mathbf{r}_1, \ldots \right)
		\arrow[r, "", mapsto] &
		\sum_{n \in \mathbb{N} }^{  } \mathbf{r}_n^{(n)} p^n
	.\end{tikzcd}
	\end{equation*}
\end{rem}


\begin{rem}[]\label{GKActionWR}
	Let's recall that in \cref{rem:GKActionR} we saw that
	the action of $\mathscr{G}_K$ extends naturally from $O_{\mathbb{C}_K}$ to $R$.
	Then, thanks to \cref{UPWittVectors}, this induces naturally
	an action of $\mathscr{G}_K$ on $W(R)$.
	More explicitly the action of $\mathscr{G}_K$ is defined by
	\begin{equation*}
		g \left( \sum_{n \in \mathbb{N} }^{  } [\mathbf{c}_n] p^n \right) =
		\sum_{n \in \mathbb{N} }^{  } [g(\mathbf{c}_n)] p^n
	,\end{equation*}
	for all $g \in \mathscr{G}_K$.
	Moreover, recalling the explicit description of the Teichmüller
	epxansion given in \cref{TeichmullerExpansionWitt}, we see that this
	action of $\mathscr{G}_K$ on $W(R)$ corresponds with the component-wise action
	(again, since it commutes with Frobenius on $R$).
\end{rem}


\begin{lem}[{\cite[Lemma 4.4.1]{Brinon}}]
	The map $\theta\colon W(R) \to O_{\mathbb{C}_K}$ is a
	$\mathscr{G}_K$-equivariant surjective ring homomorphism.
\end{lem} 


\begin{rem}[]
	Notice that $\theta$ is clearly $\mathscr{G}_K$-equivariant,
	since $\mathscr{G}_K$ acts on $O_{\mathbb{C}_K}$
	via isometries (hence via continuous maps).
	Moreover, inverting $p$, we obtain another $\mathscr{G}_K$-equivariant surjective ring homomorphism
	\begin{equation*}
	\begin{tikzcd}[row sep = 0ex
		,/tikz/column 1/.append style={anchor=base east}
		,/tikz/column 2/.append style={anchor=base west}]
		\theta_{\mathbf{Q}}\colon W(R)[1/p] \arrow[r, "", rightarrow] &
		O_{\mathbb{C}_K}[1/p] = \mathbb{C}_K
	.\end{tikzcd}
	\end{equation*} 
	It is important to notice, though, that the source ring is not a complete valuation ring.
\end{rem}


\begin{prop}[{\cite[Proposition 4.4.3]{Brinon}}]
	Let $\mathbf{p}$ be as in \cref{ExampleEltsTilt} and let
	\begin{equation*}
		\xi \coloneqq \xi_{\mathbf{p}} = [\mathbf{p}] - p \in W(R)
	.\end{equation*}
\begin{enumerate}
	\item The ideal $\ker \theta \triangleleft W(R)$ is principal and it is generated
		by $\xi$.
	\item An element $\mathbf{w} = (\mathbf{w}_0, \mathbf{w}_1, \ldots) \in \ker\theta$ generates
		the ideal if and only if $\mathbf{w}_1 \in R^{\cross}$.
\end{enumerate}
\end{prop}


\begin{cor}[{\cite[Corollary 4.4.5]{Brinon}}]
	For all $j \geq 1$ we have $W(R) \cap \left( \ker \theta_{\mathbf{Q}} \right)^j =
	\left( \ker \theta \right)^j$.
	Moreover $\bigcap_{j \geq 1} \left( \ker \theta \right)^j =
	\bigcap_{j \geq 1} \left( \ker \theta_{\mathbf{Q}} \right)^j = 0$.
\end{cor} 


\begin{rem}[]\label{kerQGKStable}
	As a consequence of the above we see that $W(R)[1/p]$
	injects into its $\ker \theta_{\mathbf{Q}}$-completion
	\begin{equation*}
	B^+_{\mathrm{dR}} \coloneqq \varprojlim_{j \geq 1}
	\frac{ W(R)[1/p] }{ \left( \ker \theta_{\mathbf{Q}} \right)^j }
	.\end{equation*}
	Recall that $\mathscr{G}_K$ acts component-wise on the ring of Witt vectors,
	so $\ker \theta_{\mathbf{Q}}$ is stable under the action of $\mathscr{G}_K$
	and the transition maps of the above projective limit are $\mathscr{G}_K$-equivariant.
	As a consequence $B^+_{\mathrm{dR}}$ inherits a natural action of $\mathscr{G}_K$
	which is compatible with that on its subring $W(R)[1/p]$.
	Then $B^+_{\mathrm{dR}}$ projects $\mathscr{G}_K$-equivariantly on its quotients
	by $\left( \ker \theta_{\mathbf{Q}} \right)^j$.
	In particular, for $j = 1$, we obtain a lift of $\theta$ 
	to a $\mathscr{G}_K$-equivariant surjection
	\begin{equation*}
	\begin{tikzcd}[row sep = 0ex
		,/tikz/column 1/.append style={anchor=base east}
		,/tikz/column 2/.append style={anchor=base west}]
		\theta^+_{\mathrm{dR}}\colon B^+_{\mathrm{dR}} \arrow[r, "", rightarrow] &
		\mathbb{C}_K
	.\end{tikzcd}
	\end{equation*} 
	Finally we see that the action of the Frobenius on $W(R)[1/p]$
	does not naturally extend to $B^+_{\mathrm{dR}}$.
	In fact $\ker \theta_{\mathbf{Q}}$ is not stable under its action
	since $\varphi(\xi) = [\mathbf{p}^p] - p \notin \ker \theta_{\mathbf{Q}}$.
\end{rem}


\noindent
By construction we saw that $B_{\mathrm{dR}}^+$ is a discrete valuation ring,
so we want to find a uniformizer with which it is nice to work with.
\begin{defn}[]
	Let $\boldsymbol\varepsilon$ be as in \cref{ExampleEltsTilt}.
	We define
	\begin{equation*}
		t \coloneqq \log \left( [\boldsymbol\varepsilon] \right) =
		\log \left( 1 + ([\boldsymbol\varepsilon] - 1)\right) =
		\sum_{n = 1 }^{ \infty } (-1)^{n+1} 
		\frac{ ([\boldsymbol\varepsilon] - 1)^n }{ n }
		\in B_{\mathrm{dR}}^+
	.\end{equation*}
\end{defn}


\begin{rem}[{\cite[pp. 60--62]{Brinon}}]\leavevmode\vspace{-\baselineskip}
\begin{enumerate}
	\item The element $[\boldsymbol\varepsilon] - 1 $ lies in
		$\ker \theta$, hence the element $\log ([\boldsymbol\varepsilon])$
		is well defined in $B_{\mathrm{dR}}^+$.

	\item One can show, via careful topological arguments, that,
		since any two different choices of $\boldsymbol\varepsilon$
		are related by $\boldsymbol\varepsilon' = \boldsymbol\varepsilon^a$
		for $a \in \mathbb{Z}_{p}^{\cross}$, then
		$t' = a t$.

	\item $\mathscr{G}_K$ acts multiplicatively on $t$ via the cyclotomic character,
		i.e. for all $g \in \mathscr{G}_K$
		\begin{equation*}
		g(t) = \chi(g) t
		\end{equation*}

	\item The element $t$ is a uniformizer of $B_{\mathrm{dR}}^+$.
\end{enumerate}
\end{rem}


\begin{defn}[Field of $p$-adic periods]
	We define the {\em field of $p$-adic periods}, also called
	the {\em de Rham period ring},
	\begin{equation*}
	B_{\mathrm{dR}} \coloneqq \mathrm{Frac}\, B_{\mathrm{dR}}^+ =
	B_{\mathrm{dR}}^+[1/t]
	.\end{equation*}
\end{defn}


\begin{rem}[]
	Notice that, just like $B_{\mathrm{dR}}^+$, the field of $p$-adic periods
	$B_{\mathrm{dR}}$ is endowed with a natural $\mathscr{G}_K$-action.
	Moreover we can notice that, set theoretically, the construction of
	$B_{\mathrm{dR}}$ depends only on $\mathbb{C}_K$ and not on $K$,
	though the choice of $K$ changes, functorially, the Galois group acting on it,
	up to restriction to a closed subgroup.
\end{rem}


\begin{prop}[{\cite[Theorem 4.4.13, example 5.1.3]{Brinon}}]
	$B_{\mathrm{dR}}$ is $\mathscr{G}_K$-regular
	(being a field) and $B_{\mathrm{dR}}^{\mathscr{G}_K} = K$.
\end{prop}


\begin{defn}[]
	We define $A_{\mathrm{cris}}$ to be the $p$-adic completion of the
	divided power envelope of $W(R)$ with respect to the ideal $\ker \theta$.
	More explicitly
	\begin{equation*}
		A_{\mathrm{cris}} = \varprojlim_{n \in \mathbb{N}} 
		\mathcal{D}_{W(R)}(\ker\theta)/p^n\mathcal{D}_{W(R)}(\ker\theta)
	.\end{equation*}
\end{defn}


\begin{rem}[{\cite[\S9.1]{Brinon}}]
	The ring $A_{\mathrm{cris}}$ is identified with a subring
	of $B_{\mathrm{dR}}^+$, whose elements are given by
	\begin{equation*}
	A_{\mathrm{cris}} = 
	\left\{ \sum_{n \in \mathbb{N} }^{  } \mathbf{a}_n \frac{ \xi^n }{ n! } \ \middle|\ 
	\mathbf{a}_n \in W(R) \text{ s.t. } \lim_{n \to \infty} \mathbf{a}_n = 0
	\text{ for the $p$-adic topology}\right\}
	.\end{equation*}
	In particular this grants that $A_{\mathrm{cris}}$ is a domain.
	Moreover the composite $A_{\mathrm{cris}} \hookrightarrow B_{\mathrm{dR}}^+ 
	\twoheadrightarrow \mathbb{C}_K$ lands in $O_{\mathbb{C}_K}$
	and induces a surjective ring homomorphism $A_{\mathrm{cris}} \twoheadrightarrow O_{\mathbb{C}_K}$.
	Also, by {\cite[Proposition 9.1.3]{Brinon}}, we see that the important
	element $t = \log ([\boldsymbol\varepsilon])$ is in $A_{\mathrm{cris}}$.
\end{rem}


\begin{rem}[]\label{AcrisAlgebraStructures}
	We can give to $A_{\mathrm{cris}}$ the structure of $W$-algebra and then also
	of $S$-algebra.
	The construction goes as follows.
	We have the injections
	\begin{equation*}
	\begin{tikzcd}
		W(R) \arrow[r, "", hookrightarrow] &
		\mathcal{D}_{W(R)}(\ker\theta) \arrow[r, "", hookrightarrow] &
		\varprojlim_{n \in \mathbb{N}} \mathcal{D}_{W(R)}(\ker\theta)/
		p^n \mathcal{D}_{W(R)}(\ker\theta) = A_{\mathrm{cris}}
	.\end{tikzcd}
	\end{equation*}
	Since, moreover, $k \hookrightarrow R$ we obtain that $W \hookrightarrow W(R)$
	and the above give a canonical $W$-algebra structure to $A_{\mathrm{cris}}$.
	We then use this structure to define the $W$-algebra morphism
	\begin{equation*}
	\begin{tikzcd}[row sep = 0ex
		,/tikz/column 1/.append style={anchor=base east}
		,/tikz/column 2/.append style={anchor=base west}]
		\alpha\colon W[u] \arrow[r, "", rightarrow] &
		A_{\mathrm{cris}} \\
		u \arrow[r, "", mapsto] & \left[ \boldsymbol\pi \right]
	,\end{tikzcd}
	\end{equation*}
	where $\boldsymbol\pi$ is defined, like $\mathbf{p}$, as
	$\boldsymbol\pi \coloneqq \big( \pi, \pi^{1/p}, \pi^{1/p^2}, \ldots \big)
	\in \varprojlim O_{\mathbb{C}_K} = R$.
	Notice, moreover, that by definition of Frobenius on both $W[u]$ and
	$A_{\mathrm{cris}}$, the map $\alpha$ is compatible with Frobenius.
	In fact the image of $\alpha$ lies in $W(R) \subset A_{\mathrm{cris}}$, hence we can check it
	here, since Frobenius of $A_{\mathrm{cris}}$ extends that of $W(R)$.
	Then, by definition we have the following square
	\begin{equation*}
	\begin{tikzcd}
		\sum_{n \in \mathbb{N} }^{  } a_n u^n
		\arrow[r, "\alpha", mapsto] \arrow[d, "\varphi"', mapsto] &
		\sum_{n \in \mathbb{N} }^{  } a_n \left[ \boldsymbol\pi \right]^n
		\arrow[d, "\varphi", mapsto] \\
		\sum_{n \in \mathbb{N} }^{  } a_n^p u^{np}
		\arrow[r, "\alpha"', mapsto] &
		\sum_{n \in \mathbb{N} }^{  } a_n^p \left[ \boldsymbol\pi \right]^{np}
	,\end{tikzcd}
	\end{equation*}
	which commutes since the Frobenius on $W(R)$ is a ring homomorphism.
	Now we notice that $\alpha(E(u)) \in \ker\theta$
	by definition of $\theta$.
	In fact $\alpha(E(u)) = E(\left[ \boldsymbol\pi \right])$
	and $\theta([\boldsymbol\pi]) = \pi$ imply that $\theta(\alpha(E(u))) = E(\pi) = 0$.
	This implies that, by universal property of divided powers envelope,
	the morphism $W[u] \to W(R) \subset A_{\mathrm{cris}}$
	induces a morphism $\mathcal{D}_{W[u]}(E(u)) \to W(R)$.
	Now we can see $W(R) \subset \mathcal{D}_{W(R)}(\ker\theta)$
	which, combined with the above comments, induce the diagram
	\begin{equation*}
	\begin{tikzcd}
		\mathcal{D}_{W[u]}(E(u)) \arrow[r, "", rightarrow] 
		\arrow[d, "", hookrightarrow] 
		\arrow[rd, "", rightarrow] &
		\mathcal{D}_{W(R)}(\ker\theta) \arrow[d, "", hookrightarrow] \\
		S \arrow[r, "", dashrightarrow] &
		A_{\mathrm{cris}}
	,\end{tikzcd}
	\end{equation*}
	where the dashed arrow exists by universal property of the completion
	of a ring with respect to the $I$-adic topology, here $I = (p)$.
\end{rem}


\begin{rem}[]
	As seen in \cref{GKActionWR}, $\mathscr{G}_K$ acts on $W(R)$ sending $pW(R)$ to $pW(R)$
	and, as seen in \cref{kerQGKStable}, $\ker\theta$ to $\ker\theta$.
	Then by universal property of divided powers envelope the action extends to
	$\mathcal{D}_{W(R)}(\ker\theta)$.
	More explicitly any $g \in \mathscr{G}_K$ induces the unique map
	\begin{equation}\label{eqn:ExtensionGKActionACris}
	\begin{tikzcd}
		&
		\mathcal{D}_{W(R)}(\ker\theta)
		\arrow[rd, "", dashrightarrow,
		start anchor=south east] & \\
		\left(W(R), \ker\theta\right) \arrow[ru, "", rightarrow,
		end anchor=south west] 
		\arrow[rr, "", rightarrow] & &
		W(R) \subset \mathcal{D}_{W(R)}(\ker\theta) 
	.\end{tikzcd}
	\end{equation}
	And this then extends to $A_{\mathrm{cris}}$.
	By {\cite[Proposition 9.1.2]{Brinon}} the action
	of $\mathscr{G}_K$ on $A_{\mathrm{cris}}$ is continuous.
	Also, following {\cite[Lemmas 9.1.7-9.1.8]{Brinon}},
	we can extend the Frobenius morphism $\varphi$ on $W(R)$ to 
	$A_{\mathrm{cris}}$.
	In fact one can check on the generator $\xi = [\mathbf{p}] - p$
	of $\ker \theta$ that $\varphi$ on $W(R)$ sends $\ker\theta + (p)$ to
	$\ker\theta + (p)$ and argue as in \cref{eqn:ExtensionGKActionACris}
	that the action extends to $\mathcal{D}_{W(R)}(\ker\theta)$
	and then to $A_{\mathrm{cris}}$.
	In particular the above holds since
	\begin{equation*}
		\varphi(\xi) = [\mathbf{p}^p] - p =
		[\mathbf{p}]^p - p =
		\underbrace{[\mathbf{p}]^p - p^p}_{\in \ker\theta} + 
		\underbrace{p^p - p}_{\in p}
	.\end{equation*}
	Moreover one can check that $\varphi(t) = pt$.
\end{rem}


\begin{defn}[]
	We denote by $B_{\mathrm{cris}}^+$ the $\mathscr{G}_K$-stable $W(R)[1/p]$ subalgebra 
	\begin{equation*}
	B_{\mathrm{cris}}^+ \coloneqq A_{\mathrm{cris}}[1/p] \subset B_{\mathrm{dR}}^+
	.\end{equation*}
	We define the {\em crystalline period ring} for $K$
	to be the $\mathscr{G}_K$-stable $W(R)[1/p]$-subalgebra of $B_{\mathrm{dR}}$
	given by $B_{\mathrm{cris}} \coloneqq B_{\mathrm{cris}}^+[1/t]$.
\end{defn}


\begin{rem}[]\leavevmode\vspace{-.2\baselineskip}
\begin{enumerate}
\item With some computations (see \cite[Proposition 9.1.3]{Brinon})
	one can show that $t^{p-1} \in p A_{\mathrm{cris}}$, hence that 
	inverting $t$ makes $p$ a unit.
	Then $B_{\mathrm{cris}} = B_{\mathrm{cris}}^+[1/t] = A_{\mathrm{cris}}[1/t]$.

\item As for $B_{\mathrm{dR}}$, set theoretically the definition of $B_{\mathrm{cris}}^+$ and
	of $B_{\mathrm{cris}}$ only depends on $\mathbb{C}_{K}$.
	Again the choice of $K$ changes, functorially, the Galois group $\mathscr{G}_K$ acting on the
	period rings.
\end{enumerate}
\end{rem}


\begin{prop}[{\cite[Proposition 9.1.6]{Brinon}}]
	$B_{\mathrm{cris}}$ is $\mathscr{G}_K$-regular
	and $B_{\mathrm{cris}}^{\mathscr{G}_K} = K_0$.
\end{prop}


\noindent
Finally we give a characterization of $A_{\mathrm{cris}}$ via
a universal property.
\begin{defn}[Formal divided power thickening]
	Let $A$ be a ring, with a principal ideal $\mathfrak{p}$
	equipped with divided powers and $V$ be an $A$-algebra.
	A {\em $\mathfrak{p}$-adic divided power thickening of $V$}
	is a surjective homomorphism of $A$-algebras
	$\theta\colon D \to V$ such that $\ker \theta$ has a divided powers structure
	compatible with those on $\mathfrak{p}$, similarly to \cref{defn:PDThickening}.
	Morphisms between divided power thickening are divided power morphisms
	making the obvious diagram commute.
	If the category of $\mathfrak{p}$-adic divided power $A$-thickening of $V$,
	whose objects and morphisms have just been defined, admits an initial object
	we call it the {\em universal $\mathfrak{p}$-adic divided power $A$-thickening of $V$}.
	
	If, moreover, $V$ is separated and complete with respect to the $\mathfrak{p}$-adic topology
	we define {\em formal $\mathfrak{p}$-adic divided power $A$-thickenings of $V$} to be
	$\mathfrak{p}$-adic divided power $A$-thickenings of $V$ which are separated and complete
	with respect to the $\mathfrak{p}$-adic topology.
	An initial object in the category of formal $\mathfrak{p}$-adic divided power $A$-thickening
	of $V$ is called universal as before.
\end{defn}


\begin{prop}[{\cite[\S2.3.2]{Fontaine}}]\label{UPACris}
	$A_{\mathrm{cris}}$ is a universal formal $p$-adic divided power $W$-thickening of
	$O_{\mathbb{C}_K}$.
\end{prop}
\begin{proof}
	It is clear that, by construction, $A_{\mathrm{cris}}$ is a
	formal $p$-adic divided power $W$-thickening of $O_{\mathbb{C}_K}$.
	In fact $\ker (A_{\mathrm{cris}} \to O_{\mathbb{C}_K})$
	is the $p$-adic completion of the P.D. ideal generated by $\ker \theta$
	in $\mathcal{D}_{W(R)}(\ker\theta)$.
	The latter is equipped with divided powers by construction, whereas
	the former by \cref{PDExtendCompletion}.
	We then need to show universality.
	Let $\left(D, \theta_D, \gamma\right)$ be another formal $p$-adic divided power
	$W$-thickening of $O_{\mathbb{C}_K}$.
	Giving a morphism of formal $p$-adic divided power $W$-thickening
	from $A_{\mathrm{cris}}$ to $D$ is equivalent to giving
	a continuous morphism
	\begin{equation*}
	\begin{tikzcd}[row sep = 0ex
		,/tikz/column 1/.append style={anchor=base east}
		,/tikz/column 2/.append style={anchor=base west}]
		\alpha\colon W(R) \arrow[r, "", rightarrow] &
		D
	\end{tikzcd}
	\end{equation*} 
	between $p$-adic rings, such that $\theta_D \circ \alpha = \left.\theta\right|_{ W(R) }$.
	This is because universal property of divided powers envelope
	allows to extend $\alpha$ uniquely to $\mathcal{D}_{W(R)}(\ker\theta)$
	and continuity to $A_{\mathrm{cris}}$.
	Let's denote $J_D \coloneqq \ker\theta$.
	We now need to notice that, given $d_1 \equiv d_2 \mod J_D$ in $D$,
	then $d_1^p \equiv d_2^p \mod pJ$.
	If $p \in J$ it is simple, since $pJ \subset J^2$ and
	it is known that $d_1^p \equiv d_2^p \mod J^2$.
	If, instead, $p \notin J$, we can see that 
	\begin{equation*}
		d_1^p - d_2^p = \left( d_1 - d_2 \right) 
		\left( d_1^{p-1} + \cdots + d_2^{p-1} \right)
	\end{equation*}
	implies that $d_1^p - d_2^p \in J$.
	Moreover, since $J$ has divided powers, we see that
	\begin{equation*}
		d_1^p - d_2^p = p! \left( \gamma_p(d_1) - \gamma_p(d_2) \right) = p \cdot y
	.\end{equation*}
	Since, moreover, $J$ is a prime ideal and $p \notin J$
	we have $y \in J$ and $d_1^p - d_2^p \in pJ$.
	Fix now $\mathbf{x} \in R$ and take, for any $m \in \mathbb{N}$, 
	$\xi_m \in D$ a lift via $\theta_D$ of $\mathbf{x}^{(m)} \in O_{\mathbb{C}_K}$, defined
	as in \cref{not:tiltingElts}.
	Then, from what we stated before, the sequence $\xi_m^{p^m}$ converges
	$p$-adically to an element $\rho(\mathbf{x}) \in D$ which does not depend on the chosen lift.
	Convergence is due to the fact that $(\mathbf{x}^{(n)})^p = \mathbf{x}^{(n-1)} \in O_{\mathbb{C}_K}$
	and the above fact, which also implies independence from the chosen lift.
%	Then we can notice that $\theta$ and $\theta_D$ are continuous in the $p$-adic topology,
%	since they are morphisms of $p$-adic rings.
	Let's recall that
	\begin{equation*}
	\theta([\mathbf{x}]) = \mathbf{x}^{(0)} =
	\lim_{n \to \infty} \widehat{\mathbf{x}_n}^{p^n}
	.\end{equation*}
	We now want construct a family of lifts
	$\xi_n$ such that $\xi_n \to \alpha([\mathbf{x}])$, in order
	to show that $\alpha([\mathbf{x}]) = \rho(\mathbf{x})$.
	We denote by $y_n \coloneqq \widehat{\mathbf{x}_n}^{p^n} \in O_{\mathbb{C}_K}$
	and by
	\begin{equation*}
		\mathbf{y}_n \coloneqq \left( y_n, y_n^{1/p}, y_n^{1/p^2}, \ldots \right) \in 
		\varprojlim_{x \mapsto x^p} O_{\mathbb{C}_K} \simeq R
	.\end{equation*}
	Then $\theta([\mathbf{y}_n]) = \widehat{\mathbf{x}_n}^{p^n}$ by definition of $\theta$.
	Moreover it is clear that $\mathbf{y}_n \to \mathbf{x}$ in $R$,
	hence that $[\mathbf{y}_n] \to [\mathbf{x}]$, if we endow $W(R)$ with the weak topology.
	The first statement is true since, by definition of the element $\mathbf{x}^{(0)}$
	(and independence of the chosen lift in its definition), we have
	$y_n \to \mathbf{x}^{(0)}$. Then an easy induction argument
	shows that $y_n^{1/p^m} \to \mathbf{x}^{(m)}$, hence that we have convergence
	in $R \simeq \varprojlim_{x \mapsto x^p}O_{\mathbb{C}_K}$ (notice that we
	can uniformly bound the distance of each component from its limit).
	By continuity of $\alpha$, then, we get that
	\begin{equation*}
		\alpha([\mathbf{x}]) = \alpha\left(\lim_{n \to \infty} [\mathbf{y}_n]\right) =
		\lim_{n \to \infty} \alpha([\mathbf{y}_n]) = \rho(\mathbf{x})
	,\end{equation*}
	where the last equality holds by compatibility of $\alpha$ with $\theta$
	which implies that $\alpha([\mathbf{y}_n])$ is a lift of $\widehat{\mathbf{x}_n}^{p^n}$.
	Then, since $\alpha$ is a continuous morphism of $p$-adic rings,
	it acts on the general element of $W(R)$ by
	\begin{equation*}
	\begin{tikzcd}[row sep = 0ex
		,/tikz/column 1/.append style={anchor=base east}
		,/tikz/column 2/.append style={anchor=base west}]
		\alpha\colon 
		(\mathbf{x}_1, \mathbf{x}_2, \ldots) =
		\sum_{n \in \mathbb{N} }^{  } [\mathbf{x}_n^{p^{-n}}] p^n
		\arrow[r, "", mapsto] &
		\sum_{n }^{  } \rho (\mathbf{x}_n^{p^{-n}}) p^n
	.\end{tikzcd}
	\end{equation*} 
	This proves uniqueness, but it is also an explicit description of $\alpha$.
	As a consequence also existence is clear, since the above $\alpha$ is
	a continuous homomorphism which commutes with the morphisms $\theta$.
\end{proof}



\subsection{Comparison morphism}
Define what $K_{\infty}$ is and the hypothesis of the thoerem, plus all the construction
of the morphism.


\begin{rem}[]
	Let $G \in \mathsf{BT}(O_K)$, seen as the inductive limit of $G_v$, 
	as in \cref{defn:pDivGroupFormalSchemes}.
	Let's notice that, in \cref{AlternativeDefnTateModule} we could have
	carried out the proof in $O_{\mathbb{C}_K}$ instead of $O_{\overline{K}}$.
	This follows from \cref{VCPBTGroups}, in which we see that
	$G_v(\overline{K}) = G_v(O_{\mathbb{C}_K})$.
	Then we have
	\begin{equation*}
		T_p(G) \simeq \mathrm{Hom}_{\mathsf{BT}(O_{\mathbb{C}_K})} 
		\big( \underline{\mathbb{Q}_p/\mathbb{Z}_{p}}, (G)_{O_{\mathbb{C}_K}} \big)
	.\end{equation*}
\end{rem}


\begin{rem}[]\label{constr:ComparisonMorphism}
	From the above remark we see that any $f \in T_p(G)$
	can be seen as a morphism of Barsotti-Tate groups
	on $O_{\mathbb{C}_K}$ between $\underline{\mathbb{Q}_p/\mathbb{Z}_{p}}$
	(base changed to $O_{\mathbb{C}_K}$) and $(G)_{O_{\mathbb{C}_K}}$.
	Since $\mathbb{D}^*$ is a contravariant functor this induces a map
	of crystals
	\begin{equation*}
	\begin{tikzcd}[row sep = 0ex
		,/tikz/column 1/.append style={anchor=base east}
		,/tikz/column 2/.append style={anchor=base west}]
		\mathbb{D}^*(f)\colon 
		\mathbb{D}^*(G_{O_{\mathbb{C}_K}})\arrow[r, "", rightarrow] &
		\mathbb{D}^*(\underline{\mathbb{Q}_p/\mathbb{Z}_{p}})
	.\end{tikzcd}
	\end{equation*} 
	Among other things \cref{PDExtendCompletion} states that
	$A_{\mathrm{cris}} \twoheadrightarrow O_{\mathbb{C}_K}$
	is a divided power thickening.
	Then we can evaluate the above at $A_{\mathrm{cris}}$ and obtain
	\begin{equation*}
	\begin{tikzcd}[row sep = 0ex
		,/tikz/column 1/.append style={anchor=base east}
		,/tikz/column 2/.append style={anchor=base west}]
		\mathbb{D}^*(f)(A_{\mathrm{cris}})\colon 
		\mathbb{D}^*(G_{O_{\mathbb{C}_K}})(A_{\mathrm{cris}})
		\arrow[r, "", rightarrow] &
		\mathbb{D}^*(\underline{\mathbb{Q}_p/\mathbb{Z}_{p}})(A_{\mathrm{cris}})
	.\end{tikzcd}
	\end{equation*} 
	Here we can notice that $\mathbb{D}^*(\underline{\mathbb{Q}_p/\mathbb{Z}_{p}})(A_{\mathrm{cris}})
	\simeq A_{\mathrm{cris}}$ (tk: I still haven't figured out why).
	Moreover one can also check that $\mathbb{D}^*(G_{O_{\mathbb{C}_K}})(A_{\mathrm{cris}}) \simeq
	A_{\mathrm{cris}} \otimes_S \mathbb{D}^*(G)(S)$ (tk: again I have not succeded),
	recalling that $A_{\mathrm{cris}}$ has an $S$-algebra structure,
	defined in \cref{AcrisAlgebraStructures}.
	Combining these observation we obtain a pairing
	\begin{equation*}
	\begin{tikzcd}[row sep = 0ex
		,/tikz/column 1/.append style={anchor=base east}
		,/tikz/column 2/.append style={anchor=base west}]
		T_p(G) \cross \mathbb{D}^*(G_{O_{\mathbb{C}_K}})(A_{\mathrm{cris}}) 
		\arrow[r, "", rightarrow] &
		A_{\mathrm{cris}} \\
		(f,a) \arrow[r, "", mapsto] &
		\mathbb{D}^*(f)(A_{\mathrm{cris}})(a)
	.\end{tikzcd}
	\end{equation*} 
	But this allows to associate to each $a \in \mathbb{D}^*(G_{O_{\mathbb{C}_K}})(A_{\mathrm{cris}})$
	the evaluation morphism
	\begin{equation*}
	\begin{tikzcd}[row sep = 0ex
		,/tikz/column 1/.append style={anchor=base east}
		,/tikz/column 2/.append style={anchor=base west}]
		\rho_a\colon T_p(G)
		\arrow[r, "", rightarrow] &
		A_{\mathrm{cris}} \\
		f \arrow[r, "", mapsto] & 
		\mathbb{D}^*(f)(A_{\mathrm{cris}})(a)
	.\end{tikzcd}
	\end{equation*} 
	Since $\mathbb{D}^*$ is an additive functor this morphism
	is $\mathbb{Z}$-linear, moreover it can be shown to be $\mathbb{Z}_{p}$-linear too.
	Finally we can notice that, set-theoretically, $T_p(G) \simeq \left( \mathbb{Z}_p \right)^h$,
	where $h$ is the height of $G$ as a $p$-divisible group.
	This means that it is a finitely generated free (hence projective) $\mathbb{Z}_{p}$-module.
	As a consequence we have a canonical isomorphism
	\begin{equation*}
	\begin{tikzcd}[row sep = 0ex
		,/tikz/column 1/.append style={anchor=base east}
		,/tikz/column 2/.append style={anchor=base west}]
		\mathrm{Hom}_{\mathbb{Z}_{p}\text{-}\mathsf{Mod}}
		\left( T_p(G), \mathbb{Z}_{p} \right) \otimes_{\mathbb{Z}_{p}} A_{\mathrm{cris}}
		\arrow[r, "\sim", rightarrow] &
		\mathrm{Hom}_{ \mathbb{Z}_{p}\text{-}\mathsf{Mod} } \left( T_p(G), A_{\mathrm{cris}} \right)
	.\end{tikzcd}
	\end{equation*} 
	All in all the above allows us to define the following
	homomorphism
	\begin{equation*}
	\begin{tikzcd}[row sep = 0ex
		,/tikz/column 1/.append style={anchor=base east}
		,/tikz/column 2/.append style={anchor=base west}]
		\rho_G\colon 
		\mathbb{D}^*(G_{O_{\mathbb{C}_K}})(A_{\mathrm{cris}})
		\arrow[r, "", rightarrow] &
		A_{\mathrm{cris}} \otimes_{\mathbb{Z}_{p}} T_p(G)^\vee
	.\end{tikzcd}
	\end{equation*} 
\end{rem}


\begin{defn}[]
	Fix $\big\{ \pi^{1/p^n} \big\}$,
	a compatible family of roots of the uniformizer $\pi$ of $K$.
	By compatible we mean that $(\pi^{1/p^n})^p = \pi^{1/p^{n-1}}$.
	We define the algebraic extension $K_{\infty}/K$ as the extension given
	by $K_\infty \coloneqq \bigcup_{n \in \mathbb{N}} K(\pi^{1/p^n})$.
	As usual we will denote by $G_{K_\infty} \coloneqq \mathrm{Gal}
	\left( \overline{K} / K_\infty \right)$ 
	the absolute Galois group of $K_\infty$.
\end{defn}


\begin{thm}[{\cite[\S6, theorem 7]{Faltings}}]
	The map constructed in \cref{constr:ComparisonMorphism}
	\begin{equation*}
	\begin{tikzcd}[row sep = 0ex
		,/tikz/column 1/.append style={anchor=base east}
		,/tikz/column 2/.append style={anchor=base west}]
		\rho_G\colon A_{\mathrm{cris}} \otimes_S \mathbb{D}^*(G)(S) 
		\arrow[r, "", rightarrow] &
		A_{\mathrm{cris}} \otimes_{\mathbb{Z}_{p}} T_p(G)^\vee
	\end{tikzcd}
	\end{equation*} 
	is a functorial injection which respects Frobenius,
	filtrations and is $G_{K_\infty}$-equivariant.
	Moreover the cokernel of $\rho_G$ is annihilated by $t$.
\end{thm}
